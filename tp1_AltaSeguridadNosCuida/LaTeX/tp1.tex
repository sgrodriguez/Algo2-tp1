\documentclass[8pt, a4paper]{article}
\usepackage[paper=a4paper, left=1cm, right=1cm, bottom=1cm, top=3cm]{geometry}
\usepackage[latin1]{inputenc}
\usepackage[T1]{fontenc}
\usepackage[spanish]{babel}
\usepackage{indentfirst}
\usepackage{fancyhdr}
\usepackage{latexsym}
\usepackage{lastpage}
\usepackage{aed2-symb,aed2-itef,aed2-tad}
\usepackage[colorlinks=true, linkcolor=blue]{hyperref}
\usepackage{calc}
\usepackage{ascii}

\newcommand{\f}[1]{\text{#1}}
\renewcommand{\paratodo}[2]{$\forall~#2$: #1}

\sloppy

\hypersetup{%
 % Para que el PDF se abra a página completa.
 pdfstartview= {FitH \hypercalcbp{\paperheight-\topmargin-1in-\headheight}},
 pdfauthor={Cátedra de Algoritmos y Estructuras de Datos II - DC - UBA},
 pdfkeywords={TADs básicos},
 pdfsubject={Tipos abstractos de datos básicos}
}

\parskip=5pt % 10pt es el tama\textbf{?}ño de fuente

% Pongo en 0 la distancia extra entre ítemes.
\let\olditemize\itemize
\def\itemize{\olditemize\itemsep=0pt}

% Acomodo fancyhdr.
\pagestyle{fancy}
\thispagestyle{fancy}
\addtolength{\headheight}{1pt}
\lhead{Algoritmos y Estructuras de Datos II}
\rhead{$2^{\mathrm{do}}$ cuatrimestre de 2015}
\cfoot{\thepage /\pageref{LastPage}}
\renewcommand{\footrulewidth}{0.4pt}

\author{Algoritmos y Estructuras de Datos II, DC, UBA.}
\date{}
\title{Trabajo Practico I, Alta Seguridad nos cuida}

\begin{document}

%Pagina de titulo e indice
\thispagestyle{empty}

\maketitle
\tableofcontents

\newpage

%TADS

%%%%%%%%%%%%%%%%%%%
% TAD UNIVERSIDAD %
%%%%%%%%%%%%%%%%%%%

\section{TAD \tadNombre{Universidad}}

\begin{tad}{\tadNombre{Universidad}}
\tadGeneros{uni}
\tadExporta{uni, Generadores, Observadores Basicos}
\tadUsa{\tadNombre{Nat, Conju($\alpha$), Bool, Tupla($\alpha_1, \ldots, \alpha_n$)}}

\tadIgualdadObservacional{u}{u'}{uni}{
alto?(u) $\igobs$  alto?(u') $\land$
ancho?(u) $\igobs$  ancho?(u') $\land$ 
obstaculos?(u) $\igobs$  obstaculos?(u') $\land$
agentes?(u) $\igobs$  agentes?(u') $\land$
estudiantes?(u) $\igobs$  estudiantes?(u') $\land$
hippies?(u) $\igobs$  hippies?(u')
}

\tadObservadores

\tadOperacion{alto?}{uni}{nat}{}
\tadOperacion{ancho?}{uni}{nat}{}
\tadOperacion{obstaculos?}{uni}{conj($pos$)}{}
\tadOperacion{agentes?}{uni}{conj($\langle$id ,\ as,\ $pos \rangle$)}{}
\tadOperacion{hippies?}{uni}{conj($\langle$id,\ pos$\rangle$)}{}
\tadOperacion{estudiantes?}{uni}{conj($\langle$id,\ est,\  $pos \rangle$)}{}

\tadGeneradores

TAD id ES Nat \
TAD hip ES Id \
TAD est ES Secu(Dir) \

\tadOperacion{nuevaUni}{conj($\langle$ $as$, $pos$ $\rangle$), nat, nat, conj($pos$)}{uni}{}

\tadOperacion{agregarE}{uni, id, est, pos}{uni}{}
\tadOperacion{agregarH}{uni, hip, pos}{uni}{}

\tadOperacion{moverAS}{uni, id}{uni}{}
\tadOperacion{moverH}{uni, id}{uni}{}
\tadOperacion{moverE}{uni, id}{uni}{}


\tadOtrasOperaciones

 \tadOperacion{capturar}{pos, conj( $\langle$ id, agente, pos $\rangle$)}{conj( $\langle$ agente, pos $\rangle$)}{}
 \tadOperacion{sancionar}{pos, conj( $\langle$ id, agente, pos $\rangle$)}{conj( $\langle$ agente, pos $\rangle$)}{}
 \tadOperacion{queTipoHay}{pos, $\langle$ cAs, cH, cEst $\rangle$}{tipo}{}
 \tadOperacion{cuatroVecinosShort}{pos, $\langle$ cAs, cH, cEst $\rangle$, uni}{multiconj(tipo)}{}
 \tadOperacion{cuatroVecinos}{pos, $\langle$ cAs, cH, cEst $\rangle$, uni, conj(pos)}{multiconj(tipo)}{}
 \tadOperacion{queSituacion}{multiconj(tipo)}{conj(situ)}{}
 \tadOperacion{damePosicionesAs}{conj( $\langle$ id, agente, pos $\rangle$)}{conj(pos)}{}

%CONFLICTOS (PROBABLEMENTE SEA SOLO LOS NOMBRES)
\tadOperacion{damePosicionesEst}{conj( $\langle$ id, agente, pos $\rangle$)}{conj(pos)}{}
\tadOperacion{damePosicionesH}{conj( $\langle$ id, pos $\rangle$)}{conj(pos)}{}

 \tadOperacion{moverEstudianteYChequearSituaciones}{id\ $is$,\ uni\ $u$}{$\langle$ cAs, cH, cEst $\rangle$}{$\exists$ $t$ $\in$ $estudiantes?(u)$ / $\Pi_{0}$($t$) $=$ $is$}
 \tadOperacion{dameEstudiante}{id\ $is$, conj( $\langle$ i, est, pos $\rangle$ )\ conjEst}{est}{$\exists$ $t$ $\in$ $conjEst$ / $\Pi_{0}$($t$) $=$ $is$}
 \tadOperacion{dameEstudiantePos}{id\ $is$, conj( $\langle$ i, est, pos $\rangle$ )\ conjEst}{pos}{$\exists$ $t$ $\in$ $conjEst$ / $\Pi_{0}$($t$) $=$ $is$}
 \tadOperacion{sacarEstudianteId}{id\ $is$, conj( $\langle$ i, est, pos $\rangle$)\ conjEst}{conj( $\langle$ i, est, pos $\rangle$}{$\exists$ $t$ $\in$ $conjEst$ / $\Pi_{0}$($t$) $=$ $is$}
 \tadOperacion{entradas?}{nat, nat}{conj(pos)}{}
 \tadOperacion{dirLibres}{pos, uni}{conj(dir)}{}
 \tadOperacion{dirValidas}{pos, nat, nat}{conj(dir)}{}
 \tadOperacion{dirNoOcupadas}{conj(dir), pos, uni}{conj(dir)}{
 	$\neg$(col?(pos) = 0 $\land$ \{o\} $\in$ cDirs) $\land$ $\neg$(fila?(pos) = 0 y \{n\} $\in$ cDirs)
 }
 \tadOperacion{posNoOcupadas}{conj(pos), uni}{conj(pos)}{}
 \tadOperacion{moverAgenteYChequearSituacion}{id $is$, uni $u$ }{$\langle$ cAs, cH, cEst $\rangle$}{$\exists$ $t$ $\in$ $agentes?(u)$ / $\Pi_{0}$($t$) $=$ $is$}
 \tadOperacion{posibleMovAs}{As , pos, uni}{conj(dir)}{}
 \tadOperacion{dameAgente}{id\ $is$, conj( $\langle$ i, as, pos $\rangle$)\ $conjAs$ }{as}{$\exists$ $t$ $\in$ $conjAs$ / $\Pi_{0}$($t$) $=$ $is$}
 \tadOperacion{dameAgentePos}{id\ $is$ ,conj( $\langle$ i, as, pos $\rangle$)\ $conjAs$}{pos}{$\exists$ $t$ $\in$ $conjAs$ / $\Pi_{0}$($t$) $=$ $is$}
 \tadOperacion{sacarAgenteId}{id\ $is$, conj( $\langle$ i, as, pos $\rangle$)\ $conjAs$}{conj( $\langle$ i, as, pos $\rangle$}{$\exists$ $t$ $\in$ $conjAs$ / $\Pi_{0}$($t$) $=$ $is$}
 \tadOperacion{moverHippieYChequearSituacion}{id\ $is$, uni, $u$}{$\langle$ cAs, cH, cEst $\rangle$}{$\exists$ $t$ $\in$ $hippies?(u)$ / $\Pi_{0}$($t$) $=$ $is$}
 \tadOperacion{posibleMovH}{pos, uni}{conj(dir)}{}
 \tadOperacion{dameHippiePos}{id\ $is$, conj( $\langle$ i, pos $\rangle$) \ $conjHip$}{pos}{$\exists$ $t$ $\in$ $conjHip$ / $\Pi_{0}$($t$) $=$ $is$}
 \tadOperacion{sacarHippieId}{id\ $is$, conj( $\langle$ i, pos $\rangle$) \ $conjHip$}{conj( $\langle$ i, pos $\rangle$)}{$\exists$ $t$ $\in$ $conjHip$ / $\Pi_{0}$($t$) $=$ $is$} 
\tadOperacion{chequearSituacionShort}{
 	 $\langle$ 
 	 	conj( $\langle$ i, as, pos $\rangle$ ), 
 	 	conj( $\langle$ i, pos $\rangle$ ), 
 	 	conj( $\langle$ i, est, pos $\rangle$ ) 
 	 $\rangle$ ,
 	 uni
}{
 	 $\langle$ 
 	 	conj( $\langle$ i, as, pos $\rangle$ ), 
 	 	conj( $\langle$ i, pos $\rangle$ ), 
 	 	conj( $\langle$ i, est, pos $\rangle$ ) 
 	 $\rangle$ ,
}{}
 \tadOperacion{chequearSituacion}{
 	 $\langle$ 
 	 	conj( $\langle$ i, as, pos $\rangle$ ), 
 	 	conj( $\langle$ i, pos $\rangle$ ), 
 	 	conj( $\langle$ i, est, pos $\rangle$ ) 
 	 $\rangle$ , 
	$\langle$ 
 	 	conj( $\langle$ i, as, pos $\rangle$ ), 
 	 	conj( $\langle$ i, pos $\rangle$ ), 
 	 	conj( $\langle$ i, est, pos $\rangle$ ) 
 	 $\rangle$ , 
 	 uni
 }{
 	 $\langle$ 
 	 	conj( $\langle$ i, est, pos $\rangle$ ), 
 	 	conj( $\langle$ i, as, pos $\rangle$ ), 
 	 	conj( $\langle$ i, est, pos $\rangle$ ) 
 	 $\rangle$ 
 }{}
 \tadOperacion{agAgenteTripla}{
 	 $\langle$ i, as, pos $\rangle$ ,
 	 $\langle$ 
 	 	conj( $\langle$ i, as, pos $\rangle$ ), 
 	 	conj( $\langle$ i, pos $\rangle$ ), 
 	 	conj( $\langle$ i, est, pos $\rangle$ ) 
 	 $\rangle$ 
 }{
 	 $\langle$ 
 	 	conj( $\langle$ i, as, pos $\rangle$ ), 
 	 	conj( $\langle$ i, pos $\rangle$ ), 
 	 	conj( $\langle$ i, est, pos $\rangle$ ) 
 	 $\rangle$ 
 }{}
 \tadOperacion{agHippieTripla}{
 	 $\langle$ i, pos $\rangle$ , 
 	 $\langle$ 
 	 	conj( $\langle$ i, as, pos $\rangle$ ), 
 	 	conj( $\langle$ i, pos $\rangle$ ), 
 	 	conj( $\langle$ i, est, pos $\rangle$ ) 
 	 $\rangle$ 
 }{
 	 $\langle$ 
 	 	conj( $\langle$ i, as, pos $\rangle$ ), 
 	 	conj( $\langle$ i, pos $\rangle$ ), 
 	 	conj( $\langle$ i, est, pos $\rangle$ ) 
 	 $\rangle$ 
 }{}
 \tadOperacion{agEstudianteTripla}{
 	 $\langle$ i, est, pos $\rangle$ , 
 	 $\langle$ 
 	 	conj( $\langle$ i, as, pos $\rangle$ ), 
 	 	conj( $\langle$ i, pos $\rangle$ ), 
 	 	conj( $\langle$ i, est, pos $\rangle$ ) 
 	 $\rangle$ 
 }{
 	 $\langle$ 
 	 	conj( $\langle$ i, as, pos $\rangle$ ), 
 	 	conj( $\langle$ i, pos $\rangle$ ), 
 	 	conj( $\langle$ i, est, pos $\rangle$ ) 
 	 $\rangle$ 
 }{}
\tadOperacion{estudiantesAdyacentesPos}{pos, conj( $\langle$ i, est, pos $\rangle$ )}{conj(est)}{}
\tadOperacion{convertirHippieAEst}{$\langle$id, pos$\rangle$, conj($\langle$id, est, pos$\rangle$)}
									{$\langle$id, est, pos$\rangle$}{}
\tadOperacion{convertirEstAHippie}{$\langle$id, est, pos$\rangle$} {$\langle$id, pos$\rangle$}{}
\tadOperacion{sacarHippieTripla}{
	$\langle$
		id, 
		pos
	$\rangle$, 
	$\langle$
		conj($\langle$i, as, pos$\rangle$), 
		conj($\langle$i, pos$\rangle$), 
		conj($\langle$i, est, pos$\rangle$)
	$\rangle$} 
	{
	$\langle$
		conj($\langle$i, as, pos$\rangle$), 
		conj($\langle$i, pos$\rangle$), 
		conj($\langle$i, est, pos$\rangle$)
	$\rangle$
}{}

\tadOperacion{dondeEstaAgente}{id\ $is$, uni \ $u$}{pos}{$\exists$ $t$ $\in$ $agentes?(u)$ / $\Pi_{0}$($t$) $=$ $is$}
\tadOperacion{dondeEstaHippie}{id\ $is$, uni \ $u$}{pos}{ $\exists$ $t$ $\in$ $hippies?(u)$ / $\Pi_{0}$($t$) $=$ $is$}
\tadOperacion{dondeEstaEstudiante}{id\ $is$, uni\ $u$}{pos}{$\exists$ $t$ $\in$ $estudiantes?(u)$ / $\Pi_{0}$($t$) $=$ $is$ }
\tadOperacion{masVigilante}{uni}{$\langle$ id, as, pos $\rangle$}{}
\tadOperacion{losMasVigilantes}{conj($\langle$ id, as, pos $\rangle$), nat}{conj($\langle$ id, as, pos $\rangle$)}{}
\tadOperacion{losMasVigilantesShort}{conj($\langle$ id, as, pos $\rangle$)}{conj($\langle$ id, as, pos $\rangle$)}{}
\tadOperacion{agenteConMenorPlaca}{conj($\langle$ id, as, pos $\rangle$), nat}{$\langle$ id, as, pos $\rangle$}{}
\tadOperacion{agenteConMenorPlacaShort}{conj($\langle$ id, as, pos $\rangle$)}{$\langle$ id, as, pos $\rangle$}{}
 
\tadOperacion{maxCapturas}{conj($\langle$ id, as, pos $\rangle$)}{nat}{}
\tadOperacion{minPlaca}{conj($\langle$ id, as, pos $\rangle$)}{nat}{}
\tadOperacion{cuantosHippies}{uni}{nat}{}
\tadOperacion{cuantosEstudiantes}{uni}{nat}{}




\tadAxiomas[
\paratodo{conj(<id, as, pos>)}{cAs}
\paratodo{conj(<id, pos>)}{cH}
\paratodo{conj(<id, est, pos>)}{cEst}
\paratodo{conj(dir)}{cDirs}
\paratodo{nat}{al, an,maxC,minN}
\paratodo{conj(pos)}{cObs}
\paratodo{uni}{u}
\paratodo{id}{i}
\paratodo{<id, est, pos>}{e}
\paratodo{<id, as, pos>}{agente}
\paratodo{<id, pos>}{h}
\paratodo{pos}{p}
\paratodo{conj(pos)}{p4v}
\paratodo{<conj(<id, as, pos>), cH conj(<id, pos>), conj(<id, est, pos>)>}{tripla, triplaInfo}
\paratodo{multiconj(tipo)}{mcT}
]

Observadores Basicos

\tadAxioma{alto? (nuevaUni(cAs, al, an, cObs))}{al}
\tadAxioma{alto? (agregarE(u, i, e, p))}{alto?($u$)}
\tadAxioma{alto? (agregarH(u, i, p))}{alto?($u$)}
\tadAxioma{alto? (moverAs(u, i))}{alto?($u$)}
\tadAxioma{alto? (moverH(u, i))}{alto?($u$)}
\tadAxioma{alto? (moverE(u, i))}{alto?($u$)}

\tadAxioma{ancho? (nuevaUni(cAs, al, an, cObs))}{an}
\tadAxioma{ancho? (agregarE(u, i, e, p))}{ancho?($u$)}
\tadAxioma{ancho? (agregarH(u, i, p))}{ancho?($u$)}
\tadAxioma{ancho? (moverAs(u, i))}{ancho?($u$)}
\tadAxioma{ancho? (moverH(u, i))}{ancho?($u$)}
\tadAxioma{ancho? (moverE(u, i))}{ancho?($u$)}

\tadAxioma{obstaculos? (nuevaUni(cAs, al, an, cObs))}{cObs}
\tadAxioma{obstaculos? (agregarE(u, i, e, p))}{obstaculos?($u$)}
\tadAxioma{obstaculos? (agregarH(u, i, p))}{obstaculos?($u$)}
\tadAxioma{obstaculos? (moverAs(u, i))}{obstaculos?($u$)}
\tadAxioma{obstaculos? (moverH(u, i))}{obstaculos?($u$)}
\tadAxioma{obstaculos? (moverE(u, i))}{obstaculos?($u$)}

\tadAxioma{agentes? (nuevaUni(cAs, al, an, cObs))}{cAs}
\tadAxioma{agentes? (agregarE(u, i, e, p))}{
	$\Pi_{1}$(
				chequearSituacionShort(
					agregarEstudianteTripla(
						$\langle$ $i$, $e$, $p$ $\rangle$, 
						\\
						$\langle$ agentes?($u$), hippies?($u$), estudiantes?($u$) $\rangle$
					)
				)
	)
}
\tadAxioma{agentes? (agregarH(u, i, pos))}{
	$\Pi_{1}$(
		chequearSituacionShort(
			agregarHippieTripla(
				$\langle$ $i$, $p$ $\rangle$, 
				\\
				$\langle$ agentes?($u$), hippies?($u$), estudiantes?($u$) $\rangle$
			)
		)
	)
}
\tadAxioma{agentes? (moverAs(u, i))}{
	$\Pi_{1}$(
		moverAgenteYSancionarYCapturar($i$, $u$)
	)
}
\tadAxioma{agentes? (moverE(u, i))}{
	$\Pi_{1}$(
		moverEstudiante($i$, $u$)
	)
}
\tadAxioma{agentes? (moverH(u, i))}{
	$\Pi_{1}$(
		moverHippie($i$,$u$)
	)
}

\tadAxioma{hippies? (nuevaUni(cAs, al, an, cObs))}{$\emptyset$}
\tadAxioma{hippies? (agregarE(u, i, e, p))}{
	$\Pi_{2}$(
		chequearSituacionShort(
			agregarEstudianteTripla(
				$\langle$ $i$, $e$, $p$ $\rangle$, 
				\\
				$\langle$ agentes?($u$), hippies?($u$), estudiantes?($u$) $\rangle$
			)
		)
	)
}
\tadAxioma{hippies? (agregarH(u, i, p))}{
	$\Pi_{2}$(
		chequearSituacionShort(
			agregarHippieTripla(
				$\langle$ $i$, $p$ $\rangle$, 
				\\
				$\langle$ agentes?($u$), hippies?($u$), estudiantes?($u$) $\rangle$
			)
		)
	)
}
\tadAxioma{hippies? (moverAs(u, i))}{
	$\Pi_{2}$(
		moverAgenteYSancionarYCapturar($i$, $u$)
	)
}
\tadAxioma{hippies? (moverE(u, i))}{
	$\Pi_{2}$(
		moverEstudiante($i$, $u$)
	)
}
\tadAxioma{hippies? (moverH(u, i))}{
	$\Pi_{2}$(
		moverHippie($i$, $u$)
	)
}

\tadAxioma{estudiantes? (nuevaUni(cAs, al, an, cObs))}{$\emptyset$}
\tadAxioma{estudiantes? (agregarE(u, i, e, p))}{
	$\Pi_{3}$(
		chequearSituacionShort(
			agregarEstudianteTripla(
				$\langle$ $i$, $e$, $p$ $\rangle$, 
				\\
				$\langle$ agentes?($u$), hippies?($u$), estudiantes?($u$) $\rangle$
			)
		)
	)
}
\tadAxioma{estudiantes? (agregarH(u, i, p))}{
	$\Pi_{3}$(
		chequearSituacionShort(
			agregarHippieTripla(
				$\langle$ $i$, $p$ $\rangle$, 
				\\
				$\langle$ agentes?($u$), hippies?($u$), estudiantes?($u$) $\rangle$
			)
		)
	)
}
\tadAxioma{estudiantes? (moverAs(u, i))}{
	$\Pi_{3}$(
		moverAgenteYSancionarYCapturar($i$, $u$)
	)
}
\tadAxioma{estudiantes? (moverE(u, i))}{
	$\Pi_{3}$(
		moverEstudiante($i$, $u$)
	)
}
\tadAxioma{estudiantes? (moverH(u, i))}{
	$\Pi_{3}$(
		moverHippie($i$, $u$)
	)
		}

Otras Operaciones

\tadAxioma{capturar(p, cAs)}{
	if ( $p$ $\in$ posiciones4Vecinas( $\Pi_{3}$( dameUno( $cAs$ ) ) ) ) 
		\\
		then
			Ag( 
				$\langle$ 
					$\Pi_{1}$($cAs$), 
					darCaptura( seg( dameUno($cAs$) ) ), 
					$\Pi_{3}$( dameUno($cAs$) ) 
				$\rangle$, 
				capturar( $p$, sinUno($cAs$) ) 
			) 
		\\
		else
			Ag( dameUno($cAs$), 
				capturar( $p$, sinUno($cAs$) ) 
			)
	\\
	fi
}

\tadAxioma{sancionar(p, cAs)}{
	if ( $p$ $\in$ posiciones4Vecinas( $\Pi_{3}$( dameUno ( cAs ) ) ) ) 
		\\
		then
			Ag( 
				$\langle$
					$\Pi_{1}$($cAs$), 
					darSancion( seg( dameUno($cAs$) ) ), 
					$\Pi_{3}$( dameUno($cAs$) )
				$\rangle$, 
				capturar( $p$, sinUno($cAs$) ) 
			)
		\\
		else
			Ag( dameUno($cAs$), sancionar( $p$, sinUno($cAs$) ) )
	\\
	fi
}

%\tadAxioma{sancionar(p, cAs)}{	
%	if ($p$ $\in$ posiciones4Vecinas( $\Pi_{3}$( dameUno($cAs$) ) ) ) 
%		\\
%		then
%			Ag( 
%				$\langle$
%					$\Pi_{1}$($cAs$), 
%					darSancion( seg( dameUno($cAs$) ) ), 
%					$\Pi_{3}$( dameUno($cAs$) )
%				>, 
%				capturar( $p$, sinUno($cAs$) ) 
%			)
%		\\
%		else
%			Ag( dameUno($cAs$), sancionar( $p$, sinUno($cAs$) ) )
%	\\
%	fi
%}

\tadAxioma{queTipoHay(p, tripla, u)}{
	if ( fila?($p$) = alto?($u$) $\lor$ \ col?($p$) = ancho?($u$) ) 
		\\
		then 
			FueraDeRango
		\\
		else
			\\
			if ( $p$ $\in$ obstaculos?($u$) ) 
				\\
				then
					Obstaculo
				\\	
				else
					\\

					% QUE ES ESTA TRIPLAAAAA??????
					
					if ($p$ $\in$ damePosicionesAs( $\Pi_{1}$(tripla) ) ) 
						\\
						then
							Agente
						\\
						else
							\\
							if ( $p$ $\in$ damePosicionesH( $\Pi_{2}$(tripla) ) ) 
								\\
								then
									Hippie
								\\
								else
									\\
									if ( $p$ $\in$ damePosicionesEst( $\Pi_{3}$(tripla) ) )
										\\
										then
											Estudiante
										\\
										else
											$\emptyset$
									\\
									fi
							\\
							fi
					\\
					fi
			\\				
			fi
	\\		
	fi
}

\tadAxioma{cuatroVecinosShort(p, tripla, u)}{
	
	% QUE ES ESTA TRIPLAAAAA??????
	
	cuatroVecinos( $p$, tripla, $u$, posiciones4Vecinas($p$) )
}

% QUE ES ESTA TRIPLAAAAA??????

\tadAxioma{cuatroVecinos(p, tripla, u, p4v)}{
	if ( vacio?($p4v$) ) 
		\\
		then
			vacio
		\\
		else
			Ag( queTipoHay( dameUno($p4v$), tripla, $u$), 
				cuatroVecinos( $p$, tripla, $u$, sinUno($p4v$)
			)
	\\
	fi
}

\tadAxioma{queSituacion(mcT)}{
	movRest?($mcT$) $\cup$ unAgente?($mcT$) $\cup$ dosHippies?($mcT$) $\cup$ cuatroEstudiantes?($mcT$)
}

\tadAxioma{damePosicionesAs(cAs)}{
	if vacio?($cAs$) 
		\\
		then
			$\emptyset$
		\\
		else
			Ag( $\Pi_{2}$( dameUno($cAs$) ), 
				damePosicionesAs( sinUno($cAs$) )
			)
	\\
	fi
}

\tadAxioma{damePosicionesEst(cEst)}{
	if vacio?($cEst$) 
		\\
		then
			$\emptyset$
		\\
		else
			Ag( $\Pi_{3}$( dameUno($cEst$) ), 
				damePosicionesEst( sinUno($cEst$) )
			)
	\\
	fi
}

\tadAxioma{damePosicionesH(cH)}{
	if vacio?($cH$) 
		\\
		then
			$\emptyset$
		\\
		else
			Ag( $\Pi_{2}$( dameUno($cH$) ), 
				damePosicionesH( sinUno($cH$) )
			)
	\\
	fi
}

\tadAxioma{moverEstudianteYChequearSituaciones(i, u)}{
	if ( vacio?( dameEstudiante( $i$, estudiantes?($u$) ) ) ) 
		\\
		then
			$\langle$
				agentes?($u$), 
				hippies?($u$), 
				estudiantes?($u$)
			$\rangle$
		\\
		else
			\\
			if $\neg$( $\Pi_{1}$( dameEstudiante($i$, $u$) ) $\in$ dirLibres( dameEstudiantePos($i$, estudiantes?($u$) ), $u$ ) ) 
				\\
				then
					$\langle$
						agentes?($u$), 
						hippies?($u$), estudiantes?($u$)
					$\rangle$
				\\
				else
					chequearSituacionShort( 
						agregarEstudianteTripla( 
							$\langle$
								$i$, 
								fin( dameEstudiante($i$, $u$) ),	
								mover( dameEstudiantePos($i$, estudiantes?($u$) ), 
								$\Pi_{1}$( dameEstudiante($i$, $u$) ) )
							$\rangle$, 
							$\langle$
								agentes?($u$), 
								hippies?($u$), 
								sacarEstudianteId( $i$, estudiantes?($u$) )
							$\rangle$ 
						)
					)
			\\
			fi
	\\
	fi
}

\tadAxioma{dameEstudiante(i, cEst)}{
	if ( $\Pi_{1}$( dameUno($cEst$) ) = $i$ ) 
		\\
		then
			$\Pi_{2}$( dameUno($cEst$) )
		\\
		else
			dameEstudiante($i$, sinUno($cEst$) )
	\\
	fi
}

\tadAxioma{dameEstudiantePos(i, cEst)}{
	if ( $\Pi_{1}$( dameUno($cEst$) ) = $i$ ) 
		\\
		then
			$\Pi_{3}$( dameUno($cEst$) )
		\\
		else
			dameEstudiantePos( $i$, sinUno($cEst$) )
	\\
	fi
}

\tadAxioma{sacarEstudianteId(i, cEst)}{
	if ( vacio?($cEst$) )
		\\
		then
			$\emptyset$
		\\
		else
			\\
			if ( $\Pi_{1}$( dameUno($cEst$) ) = $i$ )
				\\
				then
					sinUno($cEst$)
				\\
				else
					Ag( dameUno($cEst$), sacarEstudiante( $i$, sinUno($cEst$) ) )
			\\
			fi
	\\
	fi
}

\tadAxioma{entradas?(an, al)}{
	if ( $an$ = 0 )
		\\
		then
			$\emptyset$	
		\\
		else
			Ag( pos(0, $an$-1), 
				Ag( pos($al$-1, $an$-1), 
					entradas($an$-1,$al$) 
				) 
			)
	\\
	fi
}

\tadAxioma{dirLibres(p, u)}{
	dirNoOcupadas( dirValidas( $p$, ancho?($u$), alto?($u$) ), $p$, $u$ )
}

\tadAxioma{dirValidas(p, an, al)}{
	\{n,s,e,o\}
	 - (if col?($p$) = 0 then \{o\} else $\emptyset$ fi)
	 - (if col?($p$) = $an$-1 then \{e\} else $\emptyset$ fi)
	 - (if fila?($p$) = 0 then \{n\} else $\emptyset$ fi)
	 - (if fila?($p$) = $al$-1 then \{s\} else $\emptyset$ fi)
}

\tadAxioma{dirNoOcupadas( cDirs, p, u )}{
	if (vacio(cDirs)) 
		\\
		then
			$\emptyset$
		\\
		else
			\\
			if 	(	mover( $p$, dameUno($cDirs$) ) $\in$ damePosicionesH( hippies?($u$) ) $\lor$ 
					mover( $p$, dameUno($cDirs$) ) $\in$ obstaculos?($u$) $\lor$ 
					mover( $p$, dameUno($cDirs$) ) $\in$ damePosicionesAs( agentes?($u$) ) $\lor$ 
					mover( $p$, dameUno($cDirs$) ) $\in$ damePosicionesEst( estudiantes?($u$) ) 
				) 
				\\
				then\ 
					dirNoOcupadas( sinUno($cDirs$), $p$, $u$)
				\\
				else
					Ag( dameUno($cDirs$), dirNoOcupadas( sinUno($cDirs$), $p$, $u$) )
			\\
			fi
	\\
	fi
}

% Macro2 = mover( p, dameUno(cDirs) )

\tadAxioma{posNoOcupadas(cPos, u)}{
	if ( vacio?($cPos$) ) 
		\\
		then
			$\emptyset$
		\\
		else
			\\
			if $\neg$ dameUno($cPos$) $\in$ ( damePosicionesAs( agentes?($u$) ) $\cup$
											damePosicionesEst( estudiantes?($u$) ) $\cup$
											damePosicionesH( hippies?($u$) ) $\cup$
											obstaculos?($u$) 
										) 
				\\
				then
					Ag( dameUno($cPos$), posNoOcupadas(sinUno($cPos$), $u$) )
				\\
				else
					posNoOcupadas(sinUno($cPos$), $u$)
			\\
			fi
	\\
	fi
}

\tadAxioma{moverAgenteYChequearSituacion(i, u)}{
	if ( vacio?( hippies?($u$) ) $\land$ ( dameAgentePos( $i$, agentes?($u$) ) $\in$ entradas?( alto?($u$), ancho?($u$) ) ) ) $\lor$ inactivo?( dameAgente($i$, $u$) ) 
		\\
		then
			$\rangle$ agentes?($u$), hippies?($u$), estudiantes?($u$) $\langle$
		\\
		else
			\\
			if vacio?( posibleMovAs( dameAgente( $i$, agentes?($u$) ), dameAgentePos( $i$, agentes?($u$) ), $u$ ) ) 
				\\
				then
					$\rangle$ agentes?($u$), hippies?($u$), estudiantes?($u$) $\langle$
				\\
				else
					chequearSituacionShort( 
						agregarAgenteTripla( 
							$\rangle$ 
								$i$, 
								dameAgente( $i$, agentes?($u$) ), 
								mover(	dameAgentePos($i$, agentes?($u$) ), 
										dameUno(
											posibleMovAs(	
												dameAgente( $i$, agentes?($u$) ), 
												dameAgentePos( $i$, agentes?($u$) ), 
												$u$ 
											)
										)
								) 
							$\langle$,
					$\rangle$ sacarAgenteId($i$, agentes?($u$)) , hippies?($u$), estudiantes?($u$) $\langle$) )
			\\
			fi
	\\
	fi
}

\tadAxioma{posibleMovAs (agente, p, u)}{
	if ( inactivo?(agente) ) 
		\\
		then
			$\emptyset$
		\\
		else
			\\
			if ( vacio?( hippes?($u$) ) )
				\\
				then
					\\
					if ( dirLibres(agente, $u$) $\cap$ direccionesOptimas( dameUno( cPosMasCercanaShort( $p$, posNoOcupadas( entradas($an$, $al$), $u$) ) ) ) = $\emptyset$ )
						\\
						then
							dirLibres( $p$, $u$)
						\\
						else
							dirLibres( $p$, $u$) $\cap$ direccionesOptimas( dameUno( cPosMasCercanaShort($p$, posNoOcupadas( entradas($an$, $al$), $u$)) ))
					\\
					fi
			\\
			else
				\\
				if dirLibres($p$, $u$) $\cap$ direccionesOptimas( dameUno( cPosMasCercanaShort( $p$, hippies?($u$) ))) = $\emptyset$ 
					\\
					then
						dirLibres($p$, $u$)
					\\
					else
						dirLibres($p$, $u$) $\cap$ direccionesOptimas( dameUno( cPosMasCercanaShort( $p$, hippies?($u$) )))
				\\
				fi
			\\
			fi
	\\
	fi
}

\tadAxioma{dameAgente(i, cAs)}{
	if ( $\Pi_{1}$( dameUno($cAs$)) = $i$ )
		\\
		then
			$\Pi_{2}$( dameUno($cAs$) )
		\\
		else
			dameAgente( $i$, sinUno($cAs$) )
	\\
	fi
}

\tadAxioma{dameAgentePos(i, cAs)}{
	if ( $\Pi_{1}$(dameUno($cAs$)) = $i$ )
		then
			$\Pi_{3}$( dameUno($cAs$) )
		else
			dameAgente( $i$, sinUno($cAs$) )
	fi
}

\tadAxioma{sacarAgenteId(i, cAs)}{
	if ( vacio?($cAs$) )
		\\
		then
			$\emptyset$
		\\
		else
			\\
			if ( $\Pi_{1}$( dameUno($cAs$) ) = $i$ )
				\\
				then
					sinUno($cAs$)
				\\
				else
					Ag( dameUno($cAs$), sacarAgente( $i$, sinUno($cAs$) ) )
			\\
			fi
	\\
	fi
}

\tadAxioma{moverHippieYChequearSituacion(i, u)}{
	if ( vacio?( posibleMovH( dameHippiePos($i$, $u$), $u$ ) ) )
		\\
		then
			$\langle$ agentes?($u$), hippies?($u$), estudiantes?($u$) $\rangle$
		\\
		else
			chequearSituacionShort( 
				agregarHippieTripla( 
					$\langle$ 
						$i$, 
						mover( dameHippiePos($i$,$u$), dameUno( posibleMovH( dameHippiePos($i$, $u$), $u$) ) ) 
					$\rangle$, 
					$\langle$ agentes?($u$), sacarHippieId( $i$, hippies?($u$) ), estudiantes?($u$) $\rangle$
			 	) 
			)
	\\
	fi
}

\tadAxioma{posibleMovH (p, u)}{
	\IF vacio?( estudiantes?($u$) ) 
		THEN
			$\emptyset$
		ELSE
			{\IF  dirLibres($p$, $u$) $\cap$ direccionesOptimas( dameUno( cPosMasCercanaShort( $p$, damePosicionesEst( estudiantes?($u$) ) ) ) ) = $\emptyset$ 
				THEN
					dirLibres($p$, $u$)
				ELSE
					dirLibres($p$, $u$) $\cap$ direccionesOptimas( dameUno( cPosMasCercanaShort( $p$, damePosicionesEst( estudiantes?($u$) ) ) ) )
				FI}
	FI
}

\tadAxioma{dameHippiePos(i, cH)}{
	\IF prim(dameUno($cH$)) = $i$
		THEN seg(dameUno($cH$))
		ELSE dameHippiePos($i$, sinUno($cH$))
	FI
}

\tadAxioma{sacarHippieId(i, cH)}{
	\IF vacio($cH$) 
		THEN
			$\emptyset$
		ELSE
			{\IF prim(dameUno($cH$)) = i 
				THEN
					sinUno($cH$)
				ELSE
					Ag(dameUno($cH$), sacarHippie($i$, sinUno($cH$)))
			FI}
	FI
}


% QUE ES ESTA TRIPLAAAAA??????

\tadAxioma{chequearSituacionShort(tripla, u)}{
	chequearSituacion( tripla, tripla, $u$)
}

\tadAxioma{agAgenteTripla(a,  $\langle$ cAs, cH, cEst $\rangle$ )}{ $\langle$ Ag($a$, $cAs$), $cH$, $cEst$ $\rangle$ }

\tadAxioma{agHippieTripla(h,  $\langle$ cAs, cH, cEst $\rangle$ )}{ $\langle$ $cAs$, Ag($h$, $cH$), $cEst$ $\rangle$ }

\tadAxioma{agEstudianteTripla(e,  $\langle$ cAs, cH, cEst $\rangle$ )}{ $\langle$ $cAs$, $cH$, Ag($e$, $cEst$) $\rangle$ }

\tadAxioma{estudiantesAdyacentesPos(p, cEst)}{
	\IF dameEstudiantePos( dameUno($cEst$) ) $\in$ posiciones4Vecinas($p$) 
		THEN
			Ag( $\Pi_{2}$( dameUno($cEst$) ), estudiantesAdyacentesPos($p$, sinUno($cEst$) ) )
		ELSE
			estudiantesAdyacentesPos($p$, sinUno($cEst$))
	FI
}


\tadAxioma{chequearSituacion( $\langle$ cAs, cH, cEst $\rangle$ , triplaInfo, u)}{ \IF vacio?($cH$)  THEN {\IF vacio?($cEst$)  THEN   $\langle$ $cAs$, $cH$, $cEst$ $\rangle$  ELSE {\IF {movRest, unAgente} $\subseteq$ queSituacion( cuatroVecinosShort( dameEstudiantePos( dameUno($cEst$)), triplaInfo, $u$) )  THEN {\IF dosHippies $\in$ queSituacion( cuatroVecinosShort( dameEstudiantePos( dameUno($cEst$)), triplaInfo, $u$) )  THEN agregarHippieTripla(   convertirEstudianteAHippie(dameUno($cEst$)) , chequearSituacion(  $\langle$  sancionar( dameEstudiantePos( dameUno($cEst$) ), $cAs$),  $cH$,  sinUno($cEst$)   $\rangle$,  triplaInfo,  $u$ )  ) ELSE agregarEstudianteTripla(  dameUno($cEst$), chequearSituacion(  $\langle$  sancionar( dameEstudiantePos( dameUno($cEst$) ), $cAs$),  $cH$,  sinUno($cEst$)   $\rangle$ ,  triplaInfo,  $u$ )  ) FI} ELSE {\IF dosHippies $\in$ queSituacion( cuatroVecinosShort(dameEstudiantePos(dameUno($cEst$)), triplaInfo, $u$) )  THEN  agregarHippieTripla(  convertirEstudianteAHippie(dameUno($cEst$)), chequearSituacion(  $\langle$  $cAs$,  $cH$,  sinUno($cEst$)   $\rangle$,  triplaInfo,  $u$ )  ) ELSE  agregarEstudianteTripla(  dameUno($cEst$),  chequearSituacion(  $\langle$  $cAs$,  $cH$,  sinUno($cEst$) $\rangle$,  triplaInfo,  $u$ )  ) FI} FI} FI} ELSE  {\IF {movRest, unAgente} $\subseteq$ queSituacion( cuatroVecinosShort( dameHippiePos( dameUno($cH$) ), triplaInfo, $u$) )  THEN  chequearSituacion(  $\langle$  capturar(dameHippiePos(dameUno($cH$)), $cAs$),  sinUno($cH$),  $cEst$  $\rangle$,  sacarHippieTripla(dameUno($cH$), triplaInfo),  $u$ ) ELSE {\IF {movRest, cuatroEstudiantes} $\subseteq$ queSituacion( cuatroVecinosShort( dameHippiePos( dameUno($cH$) ), triplaInfo, $u$) )  THEN agregarEstudianteTripla( convertirHippieAEst(dameUno($cH$), $cEst$),  chequearSituacion(  $\langle$  $cAs$,  sinUno($cH$),  $cEst$   $\rangle$,  sacarHippieTripla(dameUno($cH$), triplaInfo),  $u$ ) ) ELSE agregarHippieTripla(  dameUno($cH$),  chequearSituacion(  $\langle$  $cAs$,  sinUno($cH$),  $cEst$   $\rangle$,  triplaInfo,  $u$)  ) FI} FI} FI} 

% %%%%%%%%%%%%%%%%%%%%%%%%%%%%%%%%%%%%%%%%% LOS COMENTARIOS DENTRO DE LOS IFS ROMPEN TODO, POR ESO LO DEJO COPIADO ACA Y TOOOOODO COMENTADO %%%%%%%%%%%%%%%%%%%%%%%%%%%%%%%%%%%%%%%%%
% \tadAxioma{chequearSituacion( $\langle$ cAs, cH, cEst $\rangle$ , triplaInfo, u)}{ 
% % % la idea de la primera tripla es hacer recursion, es la que devuelvo mod\IFicada con los cambios entre hippie y estudiantes, viceversa, y capturas/sanciones de agentes. la idea de la segunda 
% % tripla es mantener toda la informacion relevante sobre el tablero con las FIchas ya movidas por los "mover", porque sino al hacer la recursion pierdo los 4-vecinos de muchas FIchas. (ej: la ultima 
% % FIcha de la recursion no tiene nadie alrededor, nunca)
% 	\IF vacio?($cH$) 
% 		THEN
% 			{\IF vacio?($cEst$) 
% 				THEN 
% 					 $\langle$ b$cAs$, $cH$, $cEst$ $\rangle$ 
% 				ELSE % En estudiante no se tiene que mod\IFicar lo que se pasa como triplaInfo, ya que no quiero efectos en cadena (un estudiante se convierte, y luego participa, ya como hippie, en la conversion de otro estudiante)
% 					{\IF {movRest, unAgente} $\in$ queSituacion( cuatroVecinosShort( dameEstudiantePos( dameUno($cEst$)), triplaInfo, $u$) ) 
% 						THEN
% 							{\IF {dosHippies} $\in$ queSituacion( cuatroVecinosShort( dameEstudiantePos( dameUno($cEst$)), triplaInfo, $u$) ) 
% 								THEN % //sanciono al agente y agrego un hippie con el mismo ID en el lugar que estaba el Estudiante y hago recursion
% 									agregarHippieTripla(  
% 										$\langle$ 
% 											dameEstudianteId(dameUno($cEst$)), 
% 											dameEstudiantePos(dameUno($cEst$)) 
% 										$\rangle$ ,
% 										chequearSituacion( 
% 											$\langle$ 
% 												sancionar( dameEstudiantePos( dameUno($cEst$) ), $cAs$), 
% 												$cH$, 
% 												sinUno($cEst$)  
% 											$\rangle$, 
% 											triplaInfo, 
% 											$u$
% 									) 
% 								)
% 								ELSE 
% 								% //sanciono al agente y hago recursion sobre el resto
			
% 								agregarEstudianteTripla( 
% 									dameUno($cEst$),
% 									chequearSituacion( 
% 										$\langle$ 
% 											sancionar( dameEstudiantePos( dameUno($cEst$) ), $cAs$), 
% 											$cH$, 
% 											sinUno($cEst$)  
% 										$\rangle$ , 
% 										triplaInfo, 
% 										$u$) 
% 								)
% 							FI}
% 						ELSE
% 							{\IF {dosHippies} $\in$ queSituacion( cuatroVecinosShort(dameEstudiantePos(dameUno($cEst$)), triplaInfo, $u$) ) 
% 								THEN 
% 								% // agrego un hippie con el mismo ID en el lugar que estaba el Estudiante y hago recursion
% 									agregarHippieTripla( 
% 										$\langle$ 
% 											dameEstudianteId(dameUno($cEst$)), 
% 											dameEstudiantePos(dameUno($cEst$)) 
% 										$\rangle$ ,
% 										chequearSituacion( 
% 											$\langle$ 
% 												$cAs$, 
% 												$cH$, 
% 												sinUno($cEst$)  
% 											$\rangle$, 
% 											triplaInfo, 
% 											$u$
% 										) 
% 									)
% 								ELSE 
% 								% //hago recursion sin cambiar nada
			
% 								agregarEstudianteTripla( 
% 									dameUno($cEst$), 
% 									chequearSituacion( 
% 										$\langle$ 
% 											$cAs$, 
% 											$cH$, 
% 											sinUno($cEst$)
% 										$\rangle$, 
% 										triplaInfo, 
% 										$u$
% 									) 
% 								)
% 							FI}
% 					FI}
% 			FI}
% 		ELSE

% 			{\IF {movRest, unAgente} $\in$ queSituacion( cuatroVecinosShort( dameHippiePos( dameUno($cH$) ), triplaInfo, $u$) ) 
% 				THEN 
% 				% //mato al hippie (no lo agrego a la recursion), le doy una captura a los agentes cercanos y hago recursion

% 					chequearSituacion( 
% 						$\langle$ 
% 							capturar(dameHippiePos(dameUno($cH$)), $cAs$), 
% 							sinUno($cH$), 
% 							$cEst$ 
% 						$\rangle$, 
% 						$\langle$ 
% 							capturar(dameHippiePos(dameUno($cH$)), $cAs$), 
% 							sinUno($cH$), 
% 							$cEst$ 
% 						$\rangle$, 
% 						$u$
% 					)
% 				ELSE
% 					{\IF {movRest, cuatroEstudiantes} $\in$ queSituacion( cuatroVecinosShort( dameHippiePos( dameUno($cH$) ), triplaInfo, $u$) ) 
% 						THEN
% 						% //lo convierto en uno de los estudiantes que lo rodean, en este caso no me importa agregarlo a triplaInfo, porque no va a afectar a lo que le pase a los estudiantes que quedan adentro, pero si me interesa sacarlo de triplaInfo, porque no quiero que participe en convertir a un estudiante.

% 							agregarEstudianteTripla(
% 								$\langle$ 
% 									dameHippieId(dameUno($cH$)), 
% 									dameUno(estudiantesAdyacentesPos(dameHippiePos(dameUno($cH$)), $cEst$)), 
% 									dameHippiePos(dameUno($cH$)) 
% 								$\rangle$, 
% 								chequearSituacion( 
% 									$\langle$ 
% 										$cAs$, 
% 										sinUno($cH$), 
% 										$cEst$  
% 									$\rangle$, 
% 									$\langle$ 
% 										$cAs$, 
% 										sinUno($cH$), 
% 										$cEst$ 
% 									$\rangle$, 
% 									$u$
% 								)
% 							)
% 						ELSE
% 							% //no le pasa nada, hago recursion

% 							agregarHippieTripla( 
% 								dameUno($cH$), 
% 								chequearSituacion( 
% 									$\langle$ 
% 										$cAs$, 
% 										sinUno($cH$), 
% 										$cEst$  
% 									$\rangle$, 
% 								triplaInfo, 
% 								$u$) 
% 							)
% 					FI}
% 			FI}
% 	FI
% }

\tadAxioma{convertirHippieAEst(h, cEst)}{
	$\langle$
		$\Pi_{0}$($h$),
		dameUno(estudiantesAdyacentesPos($\Pi_1$($h$), $cEst$)),
		$\Pi_{1}$($h$)
	$\rangle$
}

\tadAxioma{convertirEstAHippie(e)}{
	$\langle$
		$\Pi_{0}$($e$),
		$\Pi_{2}$($e$)
	$\rangle$
}

\tadAxioma{sacarHippieTripla(h, $\langle$ cAs, cH, cEst $\rangle$)}{
	$\langle$
		$cAs$, 
		sacarHippieId($\Pi_{0}$($h$), $cH$), 
		$cEst$
	$\rangle$
}

\tadAxioma{dondeEstaAgente(i, u)}{dameAgentePos($i$, agentes?($u$))}
\tadAxioma{dondeEstaHippie(i, u)}{dameHippiePos($i$, hippies?($u$))}
\tadAxioma{dondeEstaEstudiante(i, u)}{dameEstudiantePos($i$, estudiantes?($u$))}
\tadAxioma{masVigilante(u)}{agenteConMenorPlacaShort(losMasVigilantesShort(agentes?(u)))}
\tadAxioma{losMasVigilantes(cAs, maxC)}{
        \IF vacio?($cAs$) THEN
              { $ \emptyset$}
        ELSE
                {\IF hippiesCapturados($\Pi_{1}$(dameUno($cAs$))) = maxC THEN
                                        {Ag(dameUno($cAs$), losMasVigilantes(sinUno($cAs$), maxC))}
                                ELSE
                                       { losMasVigilantes(sinUno($cAs$), maxC)}
                                FI}
        FI
}
\tadAxioma{losMasVigilantesShort(cAs)}{losMasVigilantes(cAs, maxCapturas(cAs))}
 
\tadAxioma{agenteConMenorPlaca(cAs, minN)}{
        \IF vacio?($cAs$) THEN
               { $\emptyset$}
        ELSE
                {\IF numPlaca($\Pi_{1}$(dameUno($cAs$))) = minN THEN
                                       { Ag(dameUno($cAs$), agenteConMenorPlaca(sinUno($cAs$), minN))}
                                ELSE
                                        {agenteConMenorPlaca(sinUno($cAs$), minN)}
                                FI
                        }FI
}
\tadAxioma{agenteConMenorPlacaShort(cAs)}{agenteConMenorPlaca(cAs, minPlaca(cAs))}
 
\tadAxioma{maxCapturas(cAs)}{
        \IF vacio?($cAs$) THEN
                0
        ELSE
                max(hippiesCapturados($\Pi_{1}$(dameUno($cAs$))), maxCapturas(sinUno($cAs$)))
        FI
}
\tadAxioma{minPlaca(cAs)}{
        \IF vacio?($cAs$) THEN
                0
        ELSE
                min(numPlaca($\Pi_{1}$(dameUno($cAs$))), minPlaca(sinUno($cAs$)))
        FI
}

\tadAxioma{cuantosHippies(u)}{\# hippies?(u)}
\tadAxioma{cuantosEstudiantes(u)}{\# estudiantes?(u)}


\end{tad}

%%%%%%%%%%%%%%%
% TAD AGENTES %
%%%%%%%%%%%%%%%

\section{TAD \tadNombre{Agente}}

\begin{tad}{\tadNombre{Agente}}
\tadGeneros{as}
\tadExporta{as, Generadores, Observadores Basicos, inactivo?}
\tadUsa{\tadNombre{Nat, Bool}}

\tadIgualdadObservacional{a}{a'}{as}{
	numPlaca(a) $\igobs$ numPlaca(a') $\land$
	hippiesAtrapados(a) $\igobs$ hippiesAtrapados(a') $\land$
	numSanciones(a) $\igobs$ numSanciones(a')
}

\tadObservadores

\tadOperacion{numPlaca}{as}{nat}{}
\tadOperacion{hippiesCapturados}{as}{nat}{}
\tadOperacion{numSanciones}{as}{nat}{}

\tadGeneradores

\tadOperacion{nuevoAs}{nat}{as}{}
\tadOperacion{darCaptura}{as\ a}{as}{}
\tadOperacion{darSancion}{as}{as}{}

\tadOtrasOperaciones

\tadOperacion{inactivo?}{as}{bool}{}

\tadAxiomas[
\paratodo{nat}{n}
\paratodo{as}{a}
]

Observadores Basicos

\tadAxioma{numPlaca(nuevoAs(n))}{n}
\tadAxioma{numPlaca(darCaptura(a))}{numPlaca(a)}
\tadAxioma{numPlaca(darSancion(a))}{numPlaca(a)}

\tadAxioma{hippiesCapturados(nuevoAs(n))}{0}
\tadAxioma{hippiesCapturados(darCaptura(a))}{$1 +$ hippiesCapturados(a)}
\tadAxioma{hippiesCapturados(darSancion(a))}{hippiesCapturados(a)}

\tadAxioma{numSanciones(nuevoAs(n))}{0}
\tadAxioma{numSanciones(darCaptura(a))}{numSanciones(a)}
\tadAxioma{numSanciones(darSancion(a))}{$1 +$ numSanciones(a)}

Otras Operaciones

\tadAxioma{inactivo?(a)}{\IF numSanciones(a) $>$ 3 THEN true ELSE false FI}

\end{tad}

% NICO DEFINI ESTOS GIL DE GOMA:% % % % 
%                                     %
% TAD dir es enum                     %
% TAD pos es \langle nat, nat \rangle %
% TAD est es conj(dir)                %
% % % % % % % % % % % % % % % % % % % % 

%%%%%%%%%%%%%%%%
% TAD POSICION %
%%%%%%%%%%%%%%%%


\section{TAD \tadNombre{Posicion}}


\begin{tad}{\tadNombre{Posicion}}

\tadGeneros{pos}


\tadExporta{pos, Generadores, Observadores Basicos,Otras Operaciones}

\tadUsa{\tadNombre{Nat ,Bool, Conjunto($\alpha$), Dir}}

\tadIgualdadObservacional{p}{p'}{pos}{fila?(p) $\igobs$ fila(p') $\land$ col?(p) $\igobs$ col(p') }
%%%%%%%%%%%%%%%%
% OBSERVADORES %
%%%%%%%%%%%%%%%%

\tadObservadores

\tadOperacion{fila?}{pos}{nat}{}
\tadOperacion{col?}{pos}{nat}{}

%%%%%%%%%%%%%%%%
%  GENERADORES %
%%%%%%%%%%%%%%%%

\tadGeneradores

\tadOperacion{nuevaPos}{nat\ $f$, nat\ $c$}{pos}{(f $\geq$ 0) $\land $(c $\geq$ 0)}

%%%%%%%%%%%%%%%%%%%%%
% OTRAS OPERACIONES %
%%%%%%%%%%%%%%%%%%%%%

\tadOtrasOperaciones


\tadOperacion{direccionesOptimas}{pos ,\ pos}{conj(dir)}{}
\tadOperacion{cPosMasCercana}{pos ,\ conj(pos),\ nat}{conj(pos))}{}
\tadOperacion{cPosMasCercanaShort}{pos ,\ conj(pos)}{conj(pos))}{}
\tadOperacion{menorDistancia}{pos ,\ conj(pos)\ $cPos$}{nat}{$\neg$ Vacia?(cPos)}
\tadOperacion{dist}{pos ,\ pos}{nat}{}
\tadOperacion{posiciones4Vecinas}{pos}{conj(pos)}{}
\tadOperacion{mover}{pos \ $p$ ,\ dir}{pos}{$\neg$(col?(p) = 0 \ $\land$ \ (o) = dir) \ $\land$ \ $\neg$(fila?(p) = 0 \ $\land$ \ (n) = dir)}

%%%%%%%%%%%%%%%%
%    AXIOMAS   %
%%%%%%%%%%%%%%%%

\tadAxiomas[\paratodo{pos}{p,p'}\ $\land$ \ \paratodo{nat}{f,c,menorDist} \paratodo{dir}{d}]
\tadAxioma{fila?(nuevaPos(f,c))}{f}

\tadAxioma{col?(nuevaPos(f,c))}{c}




\tadAxioma{direccionesOptimas(p,\ p')}{ \{n,s,e,o\} $-$ (\IF fila?(p) <= fila?(p') THEN \{n\} ELSE $\emptyset$ FI) \\ 
$-$ (\IF fila?(p) >= fila?(p') THEN \{s\} ELSE $\emptyset$ FI) \\
$-$ (\IF col?(p) <= col?(p') THEN \{o\} ELSE $\emptyset$ FI) \\
$-$ (\IF col?(p) >= col?(p') THEN \{e\} ELSE $\emptyset$ FI)} 


\tadAxioma{PosMasCercanaShort(p,\ cPos)}{cPosMasCercana(p,\ cPos,\ menorDistancia(p,cPos))}


 \tadAxioma{cPosMasCercana(p,\ cPos,\ menorDist)}{\IF vacio?(cPos) THEN {$\emptyset$} ELSE {\IF dist(p, dameUno(cPos)) =  menorDist THEN {Ag(p, cPosMasCercana(p, sinUno(cPos), menorDist))} ELSE {cPosMasCercana(p, sinUno(cPos), menorDist)}  FI}  FI}

\tadAxioma{menorDistancia(p,\ cPos)}{ \IF $\neg$ vacia?(cpos) THEN min(dist(p, dameUno(cPos)), menorDistancia(p, sinUno(cpos))) ELSE dist(p, dameUno(cPos)) FI}

\tadAxioma{dist(p,p')}{(max(fila?(p), fila?(p')) - min(fila?(p), fila?(p')) + (max(col?(p), col?(p')) - min(col?(p), col?(p'))}

\tadAxioma{mover(p,\ d)}{
\IF (d = n ) 
	THEN  {nuevaPos(fila?(p)-1, col?(p))}	ELSE {\IF (d=s) THEN {nuevaPos(fila?(p)+1, col?(p))} ELSE {\IF ( dir = o ) THEN {nuevaPos(fila?(p), col?(p)-1)} ELSE {nuevaPos(fila?(p), col?(p)+1)} FI} FI} FI}


\tadAxioma{posiciones4Vecinas(p)}{
	\IF (fila?(p) = 0 )\ $\land$ \ (col?(p) = 0) 
		THEN 
			{$\lbrace$ nuevaPos(1,0), nuevaPos(0,1)$\rbrace $}
		ELSE 
			{\IF fila?(p) = 0 
			 				THEN 
			 					{$\lbrace$ nuevaPos(0, col?(p)-1), nuevaPos(0, col?(p)+1), pos(1, col?(p))$\rbrace$} 
			 									ELSE 
			 					{\IF col?(p) = 0 
			 					 			 						THEN 
			 					 			 							{$\lbrace$ nuevaPos(fila?(p)-1, 0), nuevaPos(fila?(p)+1, 0), nuevaPos(fila?(p), 1)$\rbrace $}
			 					 			 						ELSE 
			 					 			 						{$\lbrace$ nuevaPos(fila?(p), col?(p)-1), nuevaPos(fila?(p), col?(p)+1),\\ nuevaPos(fila?(p)+1, col?(p)), nuevaPos(fila?(p)-1, col?(p)) $\rbrace$} FI} FI} FI		}


\end{tad}

\section{TAD \tadNombre{Direccion}}


\begin{tad}{\tadNombre{Direccion}}

\tadGeneros{dir}


\tadExporta{dir, Generadores}

\tadUsa{}

\tadIgualdadObservacionalSimple{n $\igobs$ n $\land$ s $\igobs$ s $\land$ e $\igobs$ e $\land$ o $\igobs$ o $\land$ $\neg$(s $\igobs$ n) $\land$ $\neg$(s $\igobs$ o) $\land$ $\neg$(s $\igobs$ e) $\land$ $\neg$(n $\igobs$ s) $\land$ $\neg$(n $\igobs$ e) $\land$ $\neg$(n $\igobs$ o) $\land$ $\neg$(e $\igobs$ s) $\land$ $\neg$(e $\igobs$ n) $\land$ $\neg$(e $\igobs$ o) $\land$ $\neg$(o $\igobs$ s) $\land$ \\ $\neg$(o $\igobs$ n) $\land$ $\neg$(o $\igobs$ e) }

\tadObservadores

\tadGeneradores

\tadOperacion{n}{}{dir}{}
\tadOperacion{s}{}{dir}{}
\tadOperacion{e}{}{dir}{}
\tadOperacion{o}{}{dir}{}

\tadOtrasOperaciones

\end{tad}

\section{TAD \tadNombre{SituacionesRodeo}}


\begin{tad}{\tadNombre{SituacionesRodeo}}

\tadGeneros{situ}


\tadExporta{situ, Generadores}

\tadUsa{}

\tadIgualdadObservacionalSimple{movRest $\igobs$ movRest $\land$ unAgente $\igobs$ unAgente $\land$ dosHippies $\igobs$ dosHippies $\land$ cuatroEstudiantes $\igobs$ cuatroEstudiantes $\land$ $\neg$(movRest $\igobs$ unAgente) $\land$ $\neg$(movRest $\igobs$ dosHippies) $\land$ $\neg$(movRest $\igobs$ cuatroEstudiantes) $\land$ $\neg$(unAgente $\igobs$ movRest) $\land$ $\neg$(unAgente $\igobs$ dosHippies) $\land$ $\neg$(unAgente $\igobs$ cuatroEstudiantes) $\land$ $\neg$(dosHippies $\igobs$ movRest) $\land$ $\neg$(dosHippies $\igobs$ unAgente) $\land$ $\neg$(dosHippies $\igobs$ cuatroEstudiantes) $\land$ $\neg$(cuatroEstudiantes $\igobs$ movRest) $\land$ $\neg$(cuatroEstudiantes $\igobs$ unAgente) $\land$ $\neg$(cuatroEstudiantes $\igobs$ dosHippies) }

\tadObservadores

\tadGeneradores

\tadOperacion{movRest}{}{situ}{}
\tadOperacion{unAgente}{}{situ}{}
\tadOperacion{dosHippies }{}{situ}{}
\tadOperacion{cuatroEstudiantes}{}{situ}{}

\tadOtrasOperaciones

\end{tad}

\section{TAD \tadNombre{TipoElem}}


\begin{tad}{\tadNombre{TipoElem}}

\tadGeneros{tipo}


\tadExporta{tipo, Generadores}

\tadUsa{}

\tadIgualdadObservacionalSimple{Vacio $\igobs$ Vacio \ $\land$ \ Obstaculo $\igobs$ Obstaculo $\land$ \ Agente $\igobs$ Agente $\land$ \ Hippie $\igobs$ Hippie $\land$ \ Estudiante $\igobs$ Estudiante$ \land$ \ FueraDeRango $\igobs$ FueraDeRango \ // y son diferentes de manera cruzada }

\tadObservadores

\tadGeneradores

\tadOperacion{Vacio}{}{tipo}{}
\tadOperacion{Obstaculo}{}{tipo}{}
\tadOperacion{Agente}{}{tipo}{}
\tadOperacion{Hippie}{}{tipo}{}
\tadOperacion{Estudiante}{}{tipo}{}
\tadOperacion{FueraDeRango}{}{tipo}{}

\tadOtrasOperaciones

\end{tad}

\end{document}
