\documentclass[8pt, a4paper]{article}
\usepackage[paper=a4paper, left=1.5cm, right=1.5cm, bottom=1.5cm, top=3.5cm]{geometry}
\usepackage[latin1]{inputenc}
\usepackage[T1]{fontenc}
\usepackage[spanish]{babel}
\usepackage{indentfirst}
\usepackage{fancyhdr}
\usepackage{latexsym}
\usepackage{lastpage}
\usepackage{aed2-symb,aed2-itef,aed2-tad}
\usepackage[colorlinks=true, linkcolor=blue]{hyperref}
\usepackage{calc}
\usepackage{ascii}

\newcommand{\f}[1]{\text{#1}}
\renewcommand{\paratodo}[2]{$\forall~#2$: #1}

\sloppy

\hypersetup{%
 % Para que el PDF se abra a página completa.
 pdfstartview= {FitH \hypercalcbp{\paperheight-\topmargin-1in-\headheight}},
 pdfauthor={Cátedra de Algoritmos y Estructuras de Datos II - DC - UBA},
 pdfkeywords={TADs básicos},
 pdfsubject={Tipos abstractos de datos básicos}
}

\parskip=5pt % 10pt es el tama\textbf{?}ño de fuente

% Pongo en 0 la distancia extra entre ítemes.
\let\olditemize\itemize
\def\itemize{\olditemize\itemsep=0pt}

% Acomodo fancyhdr.
\pagestyle{fancy}
\thispagestyle{fancy}
\addtolength{\headheight}{1pt}
\lhead{Algoritmos y Estructuras de Datos II}
\rhead{$2^{\mathrm{do}}$ cuatrimestre de 2015}
\cfoot{\thepage /\pageref{LastPage}}
\renewcommand{\footrulewidth}{0.4pt}

\author{Algoritmos y Estructuras de Datos II, DC, UBA.}
\date{}
\title{Trabajo Practico I, Alta Seguridad nos cuida}

\begin{document}

%Pagina de titulo e indice
\thispagestyle{empty}

\maketitle
\tableofcontents

\newpage

%TADS

%%%%%%%%%%%%%%%%%%%
% TAD UNIVERSIDAD %
%%%%%%%%%%%%%%%%%%%

\section{TAD \tadNombre{Universidad}}

\begin{tad}{\tadNombre{Universidad}}
\tadGeneros{uni}
\tadExporta{uni, Generadores, Observadores Basicos}
\tadUsa{\tadNombre{Nat, Conju($\alpha$), Bool, Tupla($\alpha_1, \ldots, \alpha_n$)}}

\tadIgualdadObservacional{u}{u'}{uni}{
alto?(u) $\igobs$  alto?(u') $\land$
ancho?(u) $\igobs$  ancho?(u') $\land$ 
obstaculos?(u) $\igobs$  obstaculos?(u') $\land$
agentes?(u) $\igobs$  agentes?(u') $\land$
estudiantes?(u) $\igobs$  estudiantes?(u') $\land$
hippies?(u) $\igobs$  hippies?(u')
}

\tadObservadores

\tadOperacion{alto?}{uni}{nat}{}
\tadOperacion{ancho?}{uni}{nat}{}
\tadOperacion{obstaculos?}{uni}{conj($pos$)}{}
\tadOperacion{agentes?}{uni}{conj($\langle as$, $pos \rangle$)}{}
\tadOperacion{hippies?}{uni}{conj($pos$)}{}
\tadOperacion{estudiantes?}{uni}{conj($\langle est$, $pos \rangle$)}{}

\tadGeneradores

//id es nat
//hip es id

\tadOperacion{nuevaUni}{conj($\langle$ $as$, $pos$ $\rangle$), nat, nat, conj($pos$)}{uni}{}

\tadOperacion{agregarE}{uni, id, est, pos}{uni}{}
\tadOperacion{agregarH}{uni, hip, pos}{uni}{}

\tadOperacion{moverAS}{uni, id}{uni}{}
\tadOperacion{moverH}{uni, id}{uni}{}
\tadOperacion{moverE}{uni, id}{uni}{}


\tadOtrasOperaciones

% \tadOperacion{capturar}{pos, conj( $\langle$ id, agente, pos $\rangle$)}{conj( $\langle$ agente, pos $\rangle$)}{}
% \tadOperacion{sancionar}{pos, conj( $\langle$ id, agente, pos $\rangle$)}{conj( $\langle$ agente, pos $\rangle$)}{}
% \tadOperacion{queTipoHay}{pos, $\langle$ cAs, cH, cEst $\rangle$}{tipoElem}{}
% \tadOperacion{cuatroVecinosShort}{pos, $\langle$ cAs, cH, cEst $\rangle$, uni}{multiconj(tipoElem)}{}
% \tadOperacion{cuatroVecinos}{pos, $\langle$ cAs, cH, cEst $\rangle$, uni, conj(pos)}{multiconj(tipoElem)}{}
% \tadOperacion{queSituacion}{multiconj(tipoElem)}{conj(structSituacionesRodeo)}{}
% \tadOperacion{damePosicionesAs}{conj( $\langle$ id, agente, pos $\rangle$)}{conj(pos)}{}
% \tadOperacion{moverEstudianteYChequearSituaciones}{id, uni}{$\langle$ cAs, cH, cEst $\rangle$}{}
% \tadOperacion{dameEstudiante}{id, conj( $\langle$ i, est, pos $\rangle$ )}{est}{i $\in$ conj}
% \tadOperacion{dameEstudiantePos}{id, conj( $\langle$ i, est, pos $\rangle$ )}{pos}{i $\in$ conj}
% \tadOperacion{sacarEstudianteId}{id, conj( $\langle$ i, est, pos $\rangle$ )}{conj( $\langle$ i, est, pos $\rangle$}{}
% \tadOperacion{entradas?}{nat, nat}{conj(pos)}{}
% \tadOperacion{dirLibres}{pos, uni}{conj(dir)}{}
% \tadOperacion{dirValidas}{pos, nat, nat}{conj(dir)}{}
% \tadOperacion{dirNoOcupadas}{conj(dir), pos, uni}{conj(dir)}{
% 	$\neg$(col?(pos) = 0 $\land$ \{o\} $\in$ cDirs) $\land$ $\neg$(fila?(pos) = 0 y \{n\} $\in$ cDirs)
% }
% \tadOperacion{posNoOcupadas}{conj(pos), uni}{conj(pos)}{}
% \tadOperacion{moverAgenteYChequearSituacion}{id, uni}{$\langle$ cAs, cH, cEst $\rangle$}{}
% \tadOperacion{posibleMovAs}{As, pos, uni}{conj(dir)}{}
% \tadOperacion{dameAgente}{conj( $\langle$ i, as, pos $\rangle$}{as}{i $\in$ conj}
% \tadOperacion{dameAgentePos}{conj( $\langle$ i, as, pos $\rangle$}{pos}{i $\in$ conj}
% \tadOperacion{sacarAgenteId}{conj( $\langle$ i, as, pos $\rangle$}{conj( $\langle$ i, as, pos $\rangle$}{}
% \tadOperacion{moverHippieYChequearSituacion}{id, uni}{$\langle$ cAs, cH, cEst $\rangle$}{}
% \tadOperacion{posibleMovH}{pos, uni}{conj(dir)}{}
% \tadOperacion{dameHippiePos}{id, conj( $\langle$ i, pos $\rangle$}{pos}{i $\in$ conj}
% \tadOperacion{sacarHippieId}{id, conj( $\langle$ i, pos $\rangle$}{conj( $\langle$ i, pos $\rangle$}{}
\tadOperacion{chequearSituacionShort(tripla, u)}{}{}{}
% \tadOperacion{chequearSituacion)}{
% 	 $\langle$ 
% 	 	conj( $\langle$ i, as, pos $\rangle$ ), 
% 	 	conj( $\langle$ i, pos $\rangle$ ), 
% 	 	conj( $\langle$ i, est, pos $\rangle$ ) 
% 	 $\rangle$ , 
% 	 uni
% }{
% 	 $\langle$ 
% 	 	conj( $\langle$ i, est, pos $\rangle$ ), 
% 	 	conj( $\langle$ i, as, pos $\rangle$ ), 
% 	 	conj( $\langle$ i, est, pos $\rangle$ ) 
% 	 $\rangle$ 
% }{}
% \tadOperacion{agAgenteTripla}{
% 	 $\langle$ i, as, pos $\rangle$ ,
% 	 $\langle$ 
% 	 	conj( $\langle$ i, as, pos $\rangle$ ), 
% 	 	conj( $\langle$ i, pos $\rangle$ ), 
% 	 	conj( $\langle$ i, est, pos $\rangle$ ) 
% 	 $\rangle$ 
% }{
% 	 $\langle$ 
% 	 	conj( $\langle$ i, as, pos $\rangle$ ), 
% 	 	conj( $\langle$ i, pos $\rangle$ ), 
% 	 	conj( $\langle$ i, est, pos $\rangle$ ) 
% 	 $\rangle$ 
% }{}
% \tadOperacion{agHippieTripla}{
% 	 $\langle$ i, pos $\rangle$ , 
% 	 $\langle$ 
% 	 	conj( $\langle$ i, as, pos $\rangle$ ), 
% 	 	conj( $\langle$ i, pos $\rangle$ ), 
% 	 	conj( $\langle$ i, est, pos $\rangle$ ) 
% 	 $\rangle$ 
% }{
% 	 $\langle$ 
% 	 	conj( $\langle$ i, as, pos $\rangle$ ), 
% 	 	conj( $\langle$ i, pos $\rangle$ ), 
% 	 	conj( $\langle$ i, est, pos $\rangle$ ) 
% 	 $\rangle$ 
% }{}
% \tadOperacion{agEstudianteTripla}{
% 	 $\langle$ i, est, pos $\rangle$ , 
% 	 $\langle$ 
% 	 	conj( $\langle$ i, as, pos $\rangle$ ), 
% 	 	conj( $\langle$ i, pos $\rangle$ ), 
% 	 	conj( $\langle$ i, est, pos $\rangle$ ) 
% 	 $\rangle$ 
% }{
% 	 $\langle$ 
% 	 	conj( $\langle$ i, as, pos $\rangle$ ), 
% 	 	conj( $\langle$ i, pos $\rangle$ ), 
% 	 	conj( $\langle$ i, est, pos $\rangle$ ) 
% 	 $\rangle$ 
% }{}
% \tadOperacion{estudiantesAdyacentesPos}{pos, conj( $\langle$ i, est, pos $\rangle$ )}{conj(est)}{}

%CONFLICTOS (PROBABLEMENTE SEA SOLO LOS NOMBRES)
\tadOperacion{damePosicionesEst}{conj( $\langle$ id, agente, pos $\rangle$)}{conj(pos)}{}
\tadOperacion{damePosicionesH}{conj( $\langle$ id, pos $\rangle$)}{conj(pos)}{}

\tadAxiomas[\paratodo{}{}]

Observadores Basicos

\tadAxioma{alto? (nuevaUni(ca, al, an, cobs))}{al}
\tadAxioma{alto? (agregarE(u, i, e, pos))}{alto?($u$)}
\tadAxioma{alto? (agregarH(u, i, pos))}{alto?($u$)}
\tadAxioma{alto? (moverAs(u, i))}{alto?($u$)}
\tadAxioma{alto? (moverH(u, i))}{alto?($u$)}
\tadAxioma{alto? (moverE(u, i))}{alto?($u$)}

\tadAxioma{ancho? (nuevaUni(ca, al, an, cobs))}{an}
\tadAxioma{ancho? (agregarE(u, i, e, pos))}{ancho?($u$)}
\tadAxioma{ancho? (agregarH(u, i, pos))}{ancho?($u$)}
\tadAxioma{ancho? (moverAs(u, i))}{ancho?($u$)}
\tadAxioma{ancho? (moverH(u, i))}{ancho?($u$)}
\tadAxioma{ancho? (moverE(u, i))}{ancho?($u$)}

\tadAxioma{obstaculos? (nuevaUni(ca, al, an, cobs))}{cobs}
\tadAxioma{obstaculos? (agregarE(u, i, e, pos))}{obstaculos?($u$)}
\tadAxioma{obstaculos? (agregarH(u, i, pos))}{obstaculos?($u$)}
\tadAxioma{obstaculos? (moverAs(u, i))}{obstaculos?($u$)}
\tadAxioma{obstaculos? (moverH(u, i))}{obstaculos?($u$)}
\tadAxioma{obstaculos? (moverE(u, i))}{obstaculos?($u$)}

\tadAxioma{agentes? (nuevaUni(ca, al, an, cobs))}{ca}
\tadAxioma{agentes? (agregarE(uni, i, e, pos))}{
	$\Pi_{1}$(
				chequearSituacionShort(
					agregarEstudianteTripla(
						$\langle$ $i$, $e$, $pos$ $\rangle$, 
						\\
						$\langle$ agentes?($u$), hippies?($u$), estudiantes?($u$) $\rangle$
					)
				)
	)
}
\tadAxioma{agentes? (agregarH(uni, i, pos))}{
	$\Pi_{1}$(
		chequearSituacionShort(
			agregarHippieTripla(
				$\langle$ $i$, $pos$ $\rangle$, 
				\\
				$\langle$ agentes?($u$), hippies?($u$), estudiantes?($u$) $\rangle$
			)
		)
	)
}
\tadAxioma{agentes? (moverAs(uni, i))}{
	$\Pi_{1}$(
		moverAgenteYSancionarYCapturar($i$, $uni$)
	)
}
\tadAxioma{agentes? (moverE(uni, i))}{
	$\Pi_{1}$(
		moverEstudiante($i$, $uni$)
	)
}
\tadAxioma{agentes? (moverH(uni, i))}{
	$\Pi_{1}$(
		moverHippie($i$,$uni$)
	)
}

\tadAxioma{hippies? (nuevaUni(ca, al, an, cobs))}{$\emptyset$}
\tadAxioma{hippies? (agregarE(uni, i, e, pos))}{
	$\Pi_{2}$(
		chequearSituacionShort(
			agregarEstudianteTripla(
				$\langle$ $i$, $e$, $pos$ $\rangle$, 
				\\
				$\langle$ agentes?($u$), hippies?($u$), estudiantes?($u$) $\rangle$
			)
		)
	)
}
\tadAxioma{hippies? (agregarH(uni, i, pos))}{
	$\Pi_{2}$(
		chequearSituacionShort(
			agregarHippieTripla(
				$\langle$ $i$, $pos$ $\rangle$, 
				\\
				$\langle$ agentes?($u$), hippies?($u$), estudiantes?($u$) $\rangle$
			)
		)
	)
}
\tadAxioma{hippies? (moverAs(uni, i))}{
	$\Pi_{2}$(
		moverAgenteYSancionarYCapturar($i$, $uni$)
	)
}
\tadAxioma{hippies? (moverE(uni, i))}{
	$\Pi_{2}$(
		moverEstudiante($i$, $uni$)
	)
}
\tadAxioma{hippies? (moverH(uni, i))}{
	$\Pi_{2}$(
		moverHippie($i$, $uni$)
	)
}

\tadAxioma{estudiantes? (nuevaUni(ca, al, an, cobs))}{$\emptyset$}
\tadAxioma{estudiantes? (agregarE(uni, i, e, pos))}{
	$\Pi_{3}$(
		chequearSituacionShort(
			agregarEstudianteTripla(
				$\langle$ $i$, $e$, $pos$ $\rangle$, 
				\\
				$\langle$ agentes?($u$), hippies?($u$), estudiantes?($u$) $\rangle$
			)
		)
	)
}
\tadAxioma{estudiantes? (agregarH(u, i, pos))}{
	$\Pi_{3}$(
		chequearSituacionShort(
			agregarHippieTripla(
				$\langle$ $i$, $pos$ $\rangle$, 
				\\
				$\langle$ agentes?($u$), hippies?($u$), estudiantes?($u$) $\rangle$
			)
		)
	)
}
\tadAxioma{estudiantes? (moverAs(uni, i))}{
	$\Pi_{3}$(
		moverAgenteYSancionarYCapturar($i$, $uni$)
	)
}
\tadAxioma{estudiantes? (moverE(uni, i))}{
	$\Pi_{3}$(
		moverEstudiante($i$, $uni$)
	)
}
\tadAxioma{estudiantes? (moverH(uni, i))}{
	$\Pi_{3}$(
		moverHippie($i$, $uni$)
	)
		}

Otras Operaciones

\tadAxioma{capturar(p, cAs)}{
	if ( $p$ $\in$ posiciones4Vecinas( $\Pi_{3}$( dameUno( $cAs$ ) ) ) ) 
		\\
		then
			Ag( 
				$\langle$ 
					$\Pi_{1}$($cAs$), 
					darCaptura( seg( dameUno($cAs$) ) ), 
					$\Pi_{3}$( dameUno($cAs$) ) 
				$\rangle$, 
				capturar( $p$, sinUno($cAs$) ) 
			) 
		\\
		else
			Ag( dameUno($cAs$), 
				capturar( $p$, sinUno($cAs$) ) 
			)
	\\
	fi
}

\tadAxioma{sancionar(p, cAs)}{
	if ( $p$ $\in$ posiciones4Vecinas( $\Pi_{3}$( dameUno ( cAs ) ) ) ) 
		\\
		then
			Ag( 
				$\langle$
					$\Pi_{1}$($cAs$), 
					darSancion( seg( dameUno($cAs$) ) ), 
					$\Pi_{3}$( dameUno($cAs$) )
				$\rangle$, 
				capturar( $p$, sinUno($cAs$) ) 
			)
		\\
		else
			Ag( dameUno($cAs$), sancionar( $p$, sinUno($cAs$) ) )
	\\
	fi
}

\tadAxioma{sancionar(p, cAs)}{	
	if ($p$ $\in$ posiciones4Vecinas( $\Pi_{3}$( dameUno($cAs$) ) ) ) 
		\\
		then
			Ag( 
				$\langle$
					$\Pi_{1}$($cAs$), 
					darSancion( seg( dameUno($cAs$) ) ), 
					$\Pi_{3}$( dameUno($cAs$) )
				>, 
				capturar( $p$, sinUno($cAs$) ) 
			)
		\\
		else
			Ag( dameUno($cAs$), sancionar( $p$, sinUno($cAs$) ) )
	\\
	fi
}

\tadAxioma{queTipoHay(p, tripla, u)}{
	if ( fila?($p$) = alto?($u$) $\lor$ \ col?($p$) = ancho?($u$) ) 
		\\
		then 
			FueraDeRango
		\\
		else
			\\
			if ( $p$ $\in$ obstaculos?($u$) ) 
				\\
				then
					Obstaculo
				\\	
				else
					\\

					% QUE ES ESTA TRIPLAAAAA??????
					
					if ($p$ $\in$ damePosicionesAs( $\Pi_{1}$(tripla) ) ) 
						\\
						then
							Agente
						\\
						else
							\\
							if ( $p$ $\in$ damePosicionesH( $\Pi_{2}$(tripla) ) ) 
								\\
								then
									Hippie
								\\
								else
									\\
									if ( $p$ $\in$ damePosicionesEst( $\Pi_{3}$(tripla) ) )
										\\
										then
											Estudiante
										\\
										else
											$\emptyset$
									\\
									fi
							\\
							fi
					\\
					fi
			\\				
			fi
	\\		
	fi
}

\tadAxioma{cuatroVecinosShort(p, tripla, u)}{
	
	% QUE ES ESTA TRIPLAAAAA??????
	
	cuatroVecinos( $p$, tripla, $u$, posiciones4Vecinas($p$) )
}

% QUE ES ESTA TRIPLAAAAA??????

\tadAxioma{cuatroVecinos(p, tripla, u, p4v)}{
	if ( vacio?($p4v$) ) 
		\\
		then
			vacio
		\\
		else
			Ag( queTipoHay( dameUno($p4v$), tripla, $u$), 
				cuatroVecinos( $p$, tripla, $u$, sinUno($p4v$)
			)
	\\
	fi
}

\tadAxioma{queSituacion(mcT)}{
	movRest?($mcT$) $\cup$ unAgente?($mcT$) $\cup$ dosHippies?($mcT$) $\cup$ cuatroEstudiantes?($mcT$)
}

\tadAxioma{damePosicionesAs(ca)}{
	if vacio?($ca$) 
		\\
		then
			$\emptyset$
		\\
		else
			Ag( $\Pi_{2}$( dameUno($ca$) ), 
				damePosicionesAs( sinUno($ca$) )
			)
	\\
	fi
}

\tadAxioma{damePosicionesEst(ce)}{
	if vacio?(ce) 
		\\
		then
			$\emptyset$
		\\
		else
			Ag( $\Pi_{3}$( dameUno($ce$) ), 
				damePosicionesEst( sinUno($ce$) )
			)
	\\
	fi
}

\tadAxioma{damePosicionesH(ch)}{
	if vacio?($ch$) 
		\\
		then
			$\emptyset$
		\\
		else
			Ag( $\Pi_{2}$( dameUno($ch$) ), 
				damePosicionesH( sinUno($ch$) )
			)
	\\
	fi
}

\tadAxioma{moverEstudianteYChequearSituaciones(i, u)}{
	if ( vacio?( dameEstudiante( $i$, estudiantes?($u$) ) ) ) 
		\\
		then
			$\langle$
				agentes?($u$), 
				hippies?($u$), 
				estudiantes?($u$)
			$\rangle$
		\\
		else
			\\
			if $\neg$( $\Pi_{1}$( dameEstudiante($i$, $u$) ) $\in$ dirLibres( dameEstudiantePos($i$, estudiantes?($u$) ), $u$ ) ) 
				\\
				then
					$\langle$
						agentes?($u$), 
						hippies?($u$), estudiantes?($u$)
					$\rangle$
				\\
				else
					chequearSituacionShort( 
						agregarEstudianteTripla( 
							$\langle$
								$i$, 
								fin( dameEstudiante($i$, $u$) ),	
								mover( dameEstudiantePos($i$, estudiantes?($u$) ), 
								$\Pi_{1}$( dameEstudiante($i$, $u$) ) )
							$\rangle$, 
							$\langle$
								agentes?($u$), 
								hippies?($u$), 
								sacarEstudianteId( $i$, estudiantes?($u$) )
							$\rangle$ 
						)
					)
			\\
			fi
	\\
	fi
}

\tadAxioma{dameEstudiante(i, cEst)}{
	if ( $\Pi_{1}$( dameUno($cEst$) ) = $i$ ) 
		\\
		then
			$\Pi_{2}$( dameUno($cEst$) )
		\\
		else
			dameEstudiante($i$, sinUno($cEst$) )
	\\
	fi
}

\tadAxioma{dameEstudiantePos(i, cEst)}{
	if ( $\Pi_{1}$( dameUno($cEst$) ) = $i$ ) 
		\\
		then
			$\Pi_{3}$( dameUno($cEst$) )
		\\
		else
			dameEstudiantePos( $i$, sinUno($cEst$) )
	\\
	fi
}

\tadAxioma{sacarEstudianteId(i, cEst)}{
	if ( vacio?($cEst$) )
		\\
		then
			$\emptyset$
		\\
		else
			\\
			if ( $\Pi_{1}$( dameUno($cEst$) ) = $i$ )
				\\
				then
					sinUno($cEst$)
				\\
				else
					Ag( dameUno($cEst$), sacarEstudiante( $i$, sinUno($cEst$) ) )
			\\
			fi
	\\
	fi
}

\tadAxioma{entradas?(an, al)}{
	if ( $an$ = 0 )
		\\
		then
			$\emptyset$	
		\\
		else
			Ag( pos(0, $an$-1), 
				Ag( pos($al$-1, $an$-1), 
					entradas($an$-1,$al$) 
				) 
			)
	\\
	fi
}

\tadAxioma{dirLibres(pos, u)}{
	dirNoOcupadas( dirValidas( $pos$, ancho?($u$), alto?($u$) ), $pos$, $u$ )
}

\tadAxioma{dirValidas(pos, an, al)}{
	\{n,s,e,o\}
	 - (if col?($pos$) = 0 then \{o\} else $\emptyset$ fi)
	 - (if col?($pos$) = $an$-1 then \{e\} else $\emptyset$ fi)
	 - (if fila?($pos$) = 0 then \{n\} else $\emptyset$ fi)
	 - (if fila?($pos$) = $al$-1 then \{s\} else $\emptyset$ fi)
}

\tadAxioma{dirNoOcupadas( cDirs, pos, u )}{
	if (vacio(cDirs)) 
		\\
		then
			$\emptyset$
		\\
		else
			\\
			if 	(	mover( $pos$, dameUno($cDirs$) ) $\in$ damePosicionesH( hippies?($u$) ) $\lor$ 
					mover( $pos$, dameUno($cDirs$) ) $\in$ obstaculos?($u$) $\lor$ 
					mover( $pos$, dameUno($cDirs$) ) $\in$ damePosicionesAs( agentes?($u$) ) $\lor$ 
					mover( $pos$, dameUno($cDirs$) ) $\in$ damePosicionesEst( estudiantes?($u$) ) 
				) 
				\\
				then\ 
					dirNoOcupadas( sinUno($cDirs$), $pos$, $u$)
				\\
				else
					Ag( dameUno($cDirs$), dirNoOcupadas( sinUno($cDirs$), $pos$, $u$) )
			\\
			fi
	\\
	fi
}

% Macro2 = mover( pos, dameUno(cDirs) )

\tadAxioma{posNoOcupadas(cPos, u)}{
	if ( vacio?($cPos$) ) 
		\\
		then
			$\emptyset$
		\\
		else
			\\
			if $\neg$ dameUno($cPos$) $\in$ ( damePosicionesAs( agentes?($u$) ) $\cup$
											damePosicionesEst( estudiantes?($u$) ) $\cup$
											damePosicionesH( hippies?($u$) ) $\cup$
											obstaculos?($u$) 
										) 
				\\
				then
					Ag( dameUno($cPos$), posNoOcupadas(sinUno($cPos$), $u$) )
				\\
				else
					posNoOcupadas(sinUno($cPos$), $u$)
			\\
			fi
	\\
	fi
}

\tadAxioma{moverAgenteYChequearSituacion(i, u)}{
	if ( vacio?( hippies?($u$) ) $\land$ ( dameAgentePos( $i$, agentes?($u$) ) $\in$ entradas?( alto?($u$), ancho?($u$) ) ) ) $\lor$ inactivo?( dameAgente($i$, $u$) ) 
		\\
		then
			$\rangle$ agentes?($u$), hippies?($u$), estudiantes?($u$) $\langle$
		\\
		else
			\\
			if vacio?( posibleMovAs( dameAgente( $i$, agentes?($u$) ), dameAgentePos( $i$, agentes?($u$) ), $u$ ) ) 
				\\
				then
					$\rangle$ agentes?($u$), hippies?($u$), estudiantes?($u$) $\langle$
				\\
				else
					chequearSituacionShort( 
						agregarAgenteTripla( 
							$\rangle$ 
								$i$, 
								dameAgente( $i$, agentes?($u$) ), 
								mover(	dameAgentePos($i$, agentes?($u$) ), 
										dameUno(
											posibleMovAs(	
												dameAgente( $i$, agentes?($u$) ), 
												dameAgentePos( $i$, agentes?($u$) ), 
												$u$ 
											)
										)
								) 
							$\langle$,
					$\rangle$ sacarAgenteId($i$, agentes?($u$)) , hippies?($u$), estudiantes?($u$) $\langle$) )
			\\
			fi
	\\
	fi
}

\tadAxioma{posibleMovAs (agente, pos, u)}{
	if ( inactivo?(agente) ) 
		\\
		then
			$\emptyset$
		\\
		else
			\\
			if ( vacio?( hippes?($u$) ) )
				\\
				then
					\\
					if ( dirLibres(agente, $u$) $\cap$ direccionesOptimas( dameUno( cPosMasCercanaShort( $pos$, posNoOcupadas( entradas($an$, $al$), $u$) ) ) ) = $\emptyset$ )
						\\
						then
							dirLibres( $pos$, $u$)
						\\
						else
							dirLibres( $pos$, $u$) $\cap$ direccionesOptimas( dameUno( cPosMasCercanaShort($pos$, posNoOcupadas( entradas($an$, $al$), $u$)) ))
					\\
					fi
			\\
			else
				\\
				if dirLibres($pos$, $u$) $\cap$ direccionesOptimas( dameUno( cPosMasCercanaShort( $pos$, hippies?($u$) ))) = $\emptyset$ 
					\\
					then
						dirLibres($pos$, $u$)
					\\
					else
						dirLibres($pos$, $u$) $\cap$ direccionesOptimas( dameUno( cPosMasCercanaShort( $pos$, hippies?($u$) )))
				\\
				fi
			\\
			fi
	\\
	fi
}

\tadAxioma{dameAgente(i, cAs)}{
	if ( $\Pi_{1}$( dameUno($cAs$)) = $i$ )
		\\
		then
			$\Pi_{2}$( dameUno($cAs$) )
		\\
		else
			dameAgente( $i$, sinUno($cAs$) )
	\\
	fi
}

\tadAxioma{dameAgentePos(i, cAs)}{
	if ( $\Pi_{1}$(dameUno($cAs$)) = $i$ )
		then
			$\Pi_{3}$( dameUno($cAs$) )
		else
			dameAgente( $i$, sinUno($cAs$) )
	fi
}

\tadAxioma{sacarAgenteId(i, cAs)}{
	if ( vacio?($cAs$) )
		\\
		then
			$\emptyset$
		\\
		else
			\\
			if ( $\Pi_{1}$( dameUno($cAs$) ) = $i$ )
				\\
				then
					sinUno($cAs$)
				\\
				else
					Ag( dameUno($cAs$), sacarAgente( $i$, sinUno($cAs$) ) )
			\\
			fi
	\\
	fi
}

\tadAxioma{moverHippieYChequearSituacion(i, u)}{
	if ( vacio?( posibleMovH( dameHippiePos($i$, $u$), $u$ ) ) )
		\\
		then
			$\langle$ agentes?($u$), hippies?($u$), estudiantes?($u$) $\rangle$
		\\
		else
			chequearSituacionShort( 
				agregarHippieTripla( 
					$\langle$ 
						$i$, 
						mover( dameHippiePos($i$,$u$), dameUno( posibleMovH( dameHippiePos($i$, $u$), $u$) ) ) 
					$\rangle$, 
					$\langle$ agentes?($u$), sacarHippieId( $i$, hippies?($u$) ), estudiantes?($u$) $\rangle$
			 	) 
			)
	\\
	fi
}

\tadAxioma{posibleMovH (pos, u)}{
	\IF vacio?( estudiantes?($u$) ) 
		THEN
			$\emptyset$
		ELSE
			{\IF  dirLibres($pos$, $u$) $\cap$ direccionesOptimas( dameUno( cPosMasCercanaShort( $pos$, damePosicionesEst( estudiantes?($u$) ) ) ) ) = $\emptyset$ 
				THEN
					dirLibres($pos$, $u$)
				ELSE
					dirLibres($pos$, $u$) $\cap$ direccionesOptimas( dameUno( cPosMasCercanaShort( $pos$, damePosicionesEst( estudiantes?($u$) ) ) ) )
				FI}
	FI
}

\tadAxioma{dameHippiePos(i, cH)}{
	\IF prim(dameUno($cH$)) = $i$
		THEN seg(dameUno($cH$))
		ELSE dameHippiePos($i$, sinUno($cH$))
	FI
}

\tadAxioma{sacarHippieId(i, cH)}{
	\IF vacio($cH$) 
		THEN
			$\emptyset$
		ELSE
			{\IF prim(dameUno($cH$)) = i 
				THEN
					sinUno($cH$)
				ELSE
					Ag(dameUno($cH$), sacarHippie($i$, sinUno($cH$)))
			FI}
	FI
}


% QUE ES ESTA TRIPLAAAAA??????

\tadAxioma{chequearSituacionShort(tripla, u)}{
	chequearSituacion( tripla, tripla, $u$)
}

\tadAxioma{agAgenteTripla(a,  $\langle$ cAs, cH, cEst $\rangle$ )}{ $\langle$ Ag($a$, $cAs$), $cH$, $cEst$ $\rangle$ }

\tadAxioma{agHippieTripla(h,  $\langle$ cAs, cH, cEst $\rangle$ )}{ $\langle$ $cAs$, Ag($h$, $cH$), $cEst$ $\rangle$ }

\tadAxioma{agEstudianteTripla(e,  $\langle$ cAs, cH, cEst $\rangle$ )}{ $\langle$ $cAs$, $cH$, Ag($e$, $cEst$) $\rangle$ }

\tadAxioma{estudiantesAdyacentesPos(p, cEst)}{
	\IF dameEstudiantePos( dameUno($cEst$) ) $\in$ posiciones4Vecinas($p$) 
		THEN
			Ag( $\Pi_{2}$( dameUno($cEst$) ), estudiantesAdyacentesPos($p$, sinUno($cEst$) ) )
		ELSE
			estudiantesAdyacentesPos($p$, sinUno($cEst$))
	FI
}


\tadAxioma{chequearSituacion( $\langle$ cAs, cH, cEst $\rangle$ , triplaInfo, u)}{
	\IF vacio?($cH$) 
		THEN
			{\IF vacio?($cEst$) 
				THEN 
					 $\langle$ $cAs$, $cH$, $cEst$ $\rangle$ 
				ELSE
					{\IF {movRest, unAgente} $\subseteq$ queSituacion( cuatroVecinosShort( dameEstudiantePos( dameUno($cEst$)), triplaInfo, $u$) ) 
						THEN
							{\IF dosHippies $\in$ queSituacion( cuatroVecinosShort( dameEstudiantePos( dameUno($cEst$)), triplaInfo, $u$) ) 
								THEN
									agregarHippieTripla(  
										convertirEstudianteAHippie(dameUno($cEst$)) ,
										chequearSituacion( 
											$\langle$ 
												sancionar( dameEstudiantePos( dameUno($cEst$) ), $cAs$), 
												$cH$, 
												sinUno($cEst$)  
											$\rangle$, 
											triplaInfo, 
											$u$
									) 
								)
								ELSE
									agregarEstudianteTripla( 
										dameUno($cEst$),
										chequearSituacion( 
											$\langle$ 
												sancionar( dameEstudiantePos( dameUno($cEst$) ), $cAs$), 
												$cH$, 
												sinUno($cEst$)  
											$\rangle$ , 
											triplaInfo, 
											$u$
										) 
								)
							FI}
						ELSE
							{\IF dosHippies $\in$ queSituacion( cuatroVecinosShort(dameEstudiantePos(dameUno($cEst$)), triplaInfo, $u$) ) 
								THEN 
									agregarHippieTripla( 
										convertirEstudianteAHippie(dameUno($cEst$)),
										chequearSituacion( 
											$\langle$ 
												$cAs$, 
												$cH$, 
												sinUno($cEst$)  
											$\rangle$, 
											triplaInfo, 
											$u$
										) 
									)
								ELSE 
									agregarEstudianteTripla( 
										dameUno($cEst$), 
										chequearSituacion( 
											$\langle$ 
												$cAs$, 
												$cH$, 
												sinUno($cEst$)
											$\rangle$, 
											triplaInfo, 
											$u$
										) 
									)
							FI}
					FI}
			FI}
		ELSE

			{\IF {movRest, unAgente} $\subseteq$ queSituacion( cuatroVecinosShort( dameHippiePos( dameUno($cH$) ), triplaInfo, $u$) ) 
				THEN 
					chequearSituacion( 
						$\langle$ 
							capturar(dameHippiePos(dameUno($cH$)), $cAs$), 
							sinUno($cH$), 
							$cEst$ 
						$\rangle$, 
						sacarHippieTripla(dameUno($cH$), triplaInfo), 
						$u$
					)
				ELSE
					{\IF {movRest, cuatroEstudiantes} $\subseteq$ queSituacion( cuatroVecinosShort( dameHippiePos( dameUno($cH$) ), triplaInfo, $u$) ) 
						THEN
							agregarEstudianteTripla(
								convertirHippieAEst(dameUno($cH$), $cEst$), 
								chequearSituacion( 
									$\langle$ 
										$cAs$, 
										sinUno($cH$), 
										$cEst$  
									$\rangle$, 
									sacarHippieTripla(dameUno($cH$), triplaInfo), 
									$u$
								)
							)
						ELSE
							agregarHippieTripla( 
								dameUno($cH$), 
								chequearSituacion( 
									$\langle$ 
										$cAs$, 
										sinUno($cH$), 
										$cEst$  
									$\rangle$, 
								triplaInfo, 
								$u$) 
							)
					FI}
			FI}
	FI
}


% %%%%%%%%%%%%%%%%%%%%%%%%%%%%%%%%%%%%%%%%% LOS COMENTARIOS DENTRO DE LOS IFS ROMPEN TODO, POR ESO LO DEJO COPIADO ACA Y TOOOOODO COMENTADO %%%%%%%%%%%%%%%%%%%%%%%%%%%%%%%%%%%%%%%%%
% \tadAxioma{chequearSituacion( $\langle$ cAs, cH, cEst $\rangle$ , triplaInfo, u)}{ 
% % % la idea de la primera tripla es hacer recursion, es la que devuelvo mod\IFicada con los cambios entre hippie y estudiantes, viceversa, y capturas/sanciones de agentes. la idea de la segunda 
% % tripla es mantener toda la informacion relevante sobre el tablero con las FIchas ya movidas por los "mover", porque sino al hacer la recursion pierdo los 4-vecinos de muchas FIchas. (ej: la ultima 
% % FIcha de la recursion no tiene nadie alrededor, nunca)
% 	\IF vacio?($cH$) 
% 		THEN
% 			{\IF vacio?($cEst$) 
% 				THEN 
% 					 $\langle$ b$cAs$, $cH$, $cEst$ $\rangle$ 
% 				ELSE % En estudiante no se tiene que mod\IFicar lo que se pasa como triplaInfo, ya que no quiero efectos en cadena (un estudiante se convierte, y luego participa, ya como hippie, en la conversion de otro estudiante)
% 					{\IF {movRest, unAgente} $\in$ queSituacion( cuatroVecinosShort( dameEstudiantePos( dameUno($cEst$)), triplaInfo, $u$) ) 
% 						THEN
% 							{\IF {dosHippies} $\in$ queSituacion( cuatroVecinosShort( dameEstudiantePos( dameUno($cEst$)), triplaInfo, $u$) ) 
% 								THEN % //sanciono al agente y agrego un hippie con el mismo ID en el lugar que estaba el Estudiante y hago recursion
% 									agregarHippieTripla(  
% 										$\langle$ 
% 											dameEstudianteId(dameUno($cEst$)), 
% 											dameEstudiantePos(dameUno($cEst$)) 
% 										$\rangle$ ,
% 										chequearSituacion( 
% 											$\langle$ 
% 												sancionar( dameEstudiantePos( dameUno($cEst$) ), $cAs$), 
% 												$cH$, 
% 												sinUno($cEst$)  
% 											$\rangle$, 
% 											triplaInfo, 
% 											$u$
% 									) 
% 								)
% 								ELSE 
% 								% //sanciono al agente y hago recursion sobre el resto
			
% 								agregarEstudianteTripla( 
% 									dameUno($cEst$),
% 									chequearSituacion( 
% 										$\langle$ 
% 											sancionar( dameEstudiantePos( dameUno($cEst$) ), $cAs$), 
% 											$cH$, 
% 											sinUno($cEst$)  
% 										$\rangle$ , 
% 										triplaInfo, 
% 										$u$) 
% 								)
% 							FI}
% 						ELSE
% 							{\IF {dosHippies} $\in$ queSituacion( cuatroVecinosShort(dameEstudiantePos(dameUno($cEst$)), triplaInfo, $u$) ) 
% 								THEN 
% 								% // agrego un hippie con el mismo ID en el lugar que estaba el Estudiante y hago recursion
% 									agregarHippieTripla( 
% 										$\langle$ 
% 											dameEstudianteId(dameUno($cEst$)), 
% 											dameEstudiantePos(dameUno($cEst$)) 
% 										$\rangle$ ,
% 										chequearSituacion( 
% 											$\langle$ 
% 												$cAs$, 
% 												$cH$, 
% 												sinUno($cEst$)  
% 											$\rangle$, 
% 											triplaInfo, 
% 											$u$
% 										) 
% 									)
% 								ELSE 
% 								% //hago recursion sin cambiar nada
			
% 								agregarEstudianteTripla( 
% 									dameUno($cEst$), 
% 									chequearSituacion( 
% 										$\langle$ 
% 											$cAs$, 
% 											$cH$, 
% 											sinUno($cEst$)
% 										$\rangle$, 
% 										triplaInfo, 
% 										$u$
% 									) 
% 								)
% 							FI}
% 					FI}
% 			FI}
% 		ELSE

% 			{\IF {movRest, unAgente} $\in$ queSituacion( cuatroVecinosShort( dameHippiePos( dameUno($cH$) ), triplaInfo, $u$) ) 
% 				THEN 
% 				% //mato al hippie (no lo agrego a la recursion), le doy una captura a los agentes cercanos y hago recursion

% 					chequearSituacion( 
% 						$\langle$ 
% 							capturar(dameHippiePos(dameUno($cH$)), $cAs$), 
% 							sinUno($cH$), 
% 							$cEst$ 
% 						$\rangle$, 
% 						$\langle$ 
% 							capturar(dameHippiePos(dameUno($cH$)), $cAs$), 
% 							sinUno($cH$), 
% 							$cEst$ 
% 						$\rangle$, 
% 						$u$
% 					)
% 				ELSE
% 					{\IF {movRest, cuatroEstudiantes} $\in$ queSituacion( cuatroVecinosShort( dameHippiePos( dameUno($cH$) ), triplaInfo, $u$) ) 
% 						THEN
% 						% //lo convierto en uno de los estudiantes que lo rodean, en este caso no me importa agregarlo a triplaInfo, porque no va a afectar a lo que le pase a los estudiantes que quedan adentro, pero si me interesa sacarlo de triplaInfo, porque no quiero que participe en convertir a un estudiante.

% 							agregarEstudianteTripla(
% 								$\langle$ 
% 									dameHippieId(dameUno($cH$)), 
% 									dameUno(estudiantesAdyacentesPos(dameHippiePos(dameUno($cH$)), $cEst$)), 
% 									dameHippiePos(dameUno($cH$)) 
% 								$\rangle$, 
% 								chequearSituacion( 
% 									$\langle$ 
% 										$cAs$, 
% 										sinUno($cH$), 
% 										$cEst$  
% 									$\rangle$, 
% 									$\langle$ 
% 										$cAs$, 
% 										sinUno($cH$), 
% 										$cEst$ 
% 									$\rangle$, 
% 									$u$
% 								)
% 							)
% 						ELSE
% 							% //no le pasa nada, hago recursion

% 							agregarHippieTripla( 
% 								dameUno($cH$), 
% 								chequearSituacion( 
% 									$\langle$ 
% 										$cAs$, 
% 										sinUno($cH$), 
% 										$cEst$  
% 									$\rangle$, 
% 								triplaInfo, 
% 								$u$) 
% 							)
% 					FI}
% 			FI}
% 	FI
% }



\end{tad}

%%%%%%%%%%%%%%%
% TAD AGENTES %
%%%%%%%%%%%%%%%

\section{TAD \tadNombre{Agente}}

\begin{tad}{\tadNombre{Agente}}
\tadGeneros{as}
\tadExporta{as, Generadores, Observadores Basicos, hippiesAtrapados}
\tadUsa{\tadNombre{Nat, Bool}}

\tadIgualdadObservacional{a}{a'}{as}{}

\tadObservadores

\tadOperacion{numPlaca}{as}{nat}{}
\tadOperacion{hippiesAtrapados}{as}{nat}{}
\tadOperacion{numSanciones}{as}{nat}{}

\tadGeneradores

\tadOperacion{nuevoAs}{nat}{as}{}
\tadOperacion{capturarH}{as\ a}{as}{}
\tadOperacion{sancionar}{as}{as}{}

\tadOtrasOperaciones

\tadOperacion{inactivo?}{as}{bool}{}

\tadAxiomas[\paratodo{}{}]

Observadores Basicos

\tadAxioma{numPlaca(nuevoAs(n))}{n}
\tadAxioma{numPlaca(capturarH(a))}{numPlaca(a)}
\tadAxioma{numPlaca(sancionar(a))}{numPlaca(a)}

\tadAxioma{hippiesAtrapados(nuevoAs(n))}{0}
\tadAxioma{hippiesAtrapados(capturarH(a))}{$1 +$ hippiesAtrapados(a)}
\tadAxioma{hippiesAtrapados(sancionar(a))}{hippiesAtrapados(a)}

\tadAxioma{numSanciones(nuevoAs(n))}{0}
\tadAxioma{numSanciones(capturarH(a))}{numSanciones(a)}
\tadAxioma{numSanciones(sancionar(a))}{$1 +$ numSanciones(a)}

Otras Operaciones

\tadAxioma{inactivo?(a)}{\IF numSanciones(a) $>$ 3 THEN true ELSE false FI}

\end{tad}

% NICO DEFINI ESTOS GIL DE GOMA:% % % % 
%                                     %
% TAD dir es enum                     %
% TAD pos es \langle nat, nat \rangle %
% TAD est es conj(dir)                %
% % % % % % % % % % % % % % % % % % % % 

%%%%%%%%%%%%%%%
% TAD AGENTES %
%%%%%%%%%%%%%%%

\section{TAD \tadNombre{Tablero}}

\begin{tad}{\tadNombre{Tablero}}
\tadGeneros{tab}
\tadExporta{tab, Generadores, Observadores Basicos}
\tadUsa{\tadNombre{Nat, Pos, Bool, Conj($\alpha$)}}

\tadIgualdadObservacional{t}{t'}{tab}{}

\tadObservadores

\tadOperacion{alto?}{tab}{nat}{}
\tadOperacion{ancho?}{tab}{nat}{}
\tadOperacion{ocupadas?}{tab}{conj($pos$)}{}

\tadGeneradores

\tadOperacion{nuevoTablero}{nat\ $an$, nat\ $al$}{tab $t$}{}
\tadOperacion{agregarFicha}{nat\ $id$, pos\ $p$}{tab $t$}{}

\tadAxiomas[\paratodo{}{}]

\tadAxioma{}{}

\end{tad}


\end{document}
